% Options for packages loaded elsewhere
\PassOptionsToPackage{unicode}{hyperref}
\PassOptionsToPackage{hyphens}{url}
%
\documentclass[
  11pt,
]{article}
\usepackage{lmodern}
\usepackage{amssymb,amsmath}
\usepackage{ifxetex,ifluatex}
\ifnum 0\ifxetex 1\fi\ifluatex 1\fi=0 % if pdftex
  \usepackage[T1]{fontenc}
  \usepackage[utf8]{inputenc}
  \usepackage{textcomp} % provide euro and other symbols
\else % if luatex or xetex
  \usepackage{unicode-math}
  \defaultfontfeatures{Scale=MatchLowercase}
  \defaultfontfeatures[\rmfamily]{Ligatures=TeX,Scale=1}
\fi
% Use upquote if available, for straight quotes in verbatim environments
\IfFileExists{upquote.sty}{\usepackage{upquote}}{}
\IfFileExists{microtype.sty}{% use microtype if available
  \usepackage[]{microtype}
  \UseMicrotypeSet[protrusion]{basicmath} % disable protrusion for tt fonts
}{}
\makeatletter
\@ifundefined{KOMAClassName}{% if non-KOMA class
  \IfFileExists{parskip.sty}{%
    \usepackage{parskip}
  }{% else
    \setlength{\parindent}{0pt}
    \setlength{\parskip}{6pt plus 2pt minus 1pt}}
}{% if KOMA class
  \KOMAoptions{parskip=half}}
\makeatother
\usepackage{xcolor}
\IfFileExists{xurl.sty}{\usepackage{xurl}}{} % add URL line breaks if available
\IfFileExists{bookmark.sty}{\usepackage{bookmark}}{\usepackage{hyperref}}
\hypersetup{
  pdftitle={A framework for effective application of machine learning to microbiome-based classification problems},
  hidelinks,
  pdfcreator={LaTeX via pandoc}}
\urlstyle{same} % disable monospaced font for URLs
\usepackage[margin=1in]{geometry}
\usepackage{graphicx,grffile}
\makeatletter
\def\maxwidth{\ifdim\Gin@nat@width>\linewidth\linewidth\else\Gin@nat@width\fi}
\def\maxheight{\ifdim\Gin@nat@height>\textheight\textheight\else\Gin@nat@height\fi}
\makeatother
% Scale images if necessary, so that they will not overflow the page
% margins by default, and it is still possible to overwrite the defaults
% using explicit options in \includegraphics[width, height, ...]{}
\setkeys{Gin}{width=\maxwidth,height=\maxheight,keepaspectratio}
% Set default figure placement to htbp
\makeatletter
\def\fps@figure{htbp}
\makeatother
\setlength{\emergencystretch}{3em} % prevent overfull lines
\providecommand{\tightlist}{%
  \setlength{\itemsep}{0pt}\setlength{\parskip}{0pt}}
\setcounter{secnumdepth}{-\maxdimen} % remove section numbering
\usepackage{booktabs}
\usepackage{longtable}
\usepackage{array}
\usepackage{multirow}
\usepackage{wrapfig}
\usepackage{float}
\usepackage{colortbl}
\usepackage{pdflscape}
\usepackage{tabu}
\usepackage{threeparttable}
\usepackage{threeparttablex}
\usepackage[normalem]{ulem}
\usepackage{makecell}
\usepackage{caption}
\usepackage{hyperref}
\usepackage{helvet} % Helvetica font
\renewcommand*\familydefault{\sfdefault} % Use the sans serif version of the font
\usepackage[T1]{fontenc}
\usepackage[labelfont=bf]{caption}

\usepackage[none]{hyphenat}

\usepackage{setspace}
\doublespacing
\setlength{\parskip}{1em}

\usepackage{lineno}

\usepackage{pdfpages}
\floatplacement{figure}{H} % Keep the figure up top of the page

\title{\textbf{A framework for effective application of machine learning to
microbiome-based classification problems}}
\author{}
\date{\vspace{-2.5em}}

\begin{document}
\maketitle

\vspace{30mm}

Running title: Machine learning framework to model microbiome data

\vspace{20mm}

Begüm D. Topçuoğlu\({^1}\), Nicholas A. Lesniak\({^1}\), Mack
Ruffin\({^3}\), Jenna Wiens\textsuperscript{2\(\dagger\)}, Patrick D.
Schloss\textsuperscript{1\(\dagger\)}

\vspace{30mm}

\(\dagger\) To whom correspondence should be addressed:
\href{mailto:pschloss@umich.edu}{\nolinkurl{pschloss@umich.edu}},
\href{mailto:wiensj@umich.edu}{\nolinkurl{wiensj@umich.edu}}

1. Department of Microbiology and Immunology, University of Michigan,
Ann Arbor, MI 48109

2. Department of Electrical Engineering and Computer Science, University
of Michigan, Ann Arbor, MI 48109

3. Department of Family Medicine and Community Medicine, Penn State
Hershey Medical Center, Hershey, PA

\newpage
\linenumbers

\hypertarget{abstract}{%
\subsection{Abstract}\label{abstract}}

Machine learning (ML) modeling of the human microbiome has the potential
to identify microbial biomarkers and aid in the diagnosis of many
diseases such as inflammatory bowel disease, diabetes, and colorectal
cancer. Progress has been made towards developing ML models that predict
health outcomes using bacterial abundances, but inconsistent adoption of
training and evaluation methods call the validity of these models into
question. Furthermore, there appears to be a preference by many
researchers to favor increased model complexity over interpretability.
To overcome these challenges, we trained seven models that used fecal
16S rRNA sequence data to predict the presence of colonic screen
relevant neoplasias (SRNs; n=490 patients, 261 controls and 229 cases).
We developed a reusable open-source pipeline to train, validate, and
interpret ML models. To show the effect of model selection, we assessed
the predictive performance, interpretability, and training time of
L2-regularized logistic regression, L1 and L2-regularized support vector
machines (SVM) with linear and radial basis function kernels, decision
trees, random forest, and gradient boosted trees (XGBoost). The random
forest model performed best at detecting SRNs with an AUROC of 0.695
{[}IQR 0.651-0.739{]} but was slow to train (83.2 h) and not inherently
interpretable. Despite its simplicity, L2-regularized logistic
regression followed random forest in predictive performance with an
AUROC of 0.680 {[}IQR 0.625-0.735{]}, trained faster (12 min), and was
inherently interpretable. Our analysis highlights the importance of
choosing an ML approach based on the goal of the study, as the choice
will inform expectations of performance and interpretability.

\newpage

\hypertarget{importance}{%
\subsection{Importance}\label{importance}}

Diagnosing diseases using machine learning (ML) is rapidly being adopted
in microbiome studies. However, the estimated performance associated
with these models is likely over-optimistic. Moreover, there is a trend
towards using black box models without a discussion of the difficulty of
interpreting such models when trying to identify microbial biomarkers of
disease. This work represents a step towards developing more
reproducible ML practices in applying ML to microbiome research. We
implement a rigorous pipeline and emphasize the importance of selecting
ML models that reflect the goal of the study. These concepts are not
particular to the study of human health but can also be applied to
environmental microbiology studies.

\newpage

\hypertarget{background}{%
\subsection{Background}\label{background}}

As the number of people represented in human microbiome datasets grow,
there is an increasing desire to use microbiome data to diagnose
diseases. However, the structure of the human microbiome is remarkably
variable among individuals to the point where it is often difficult to
identify the bacterial populations that are associated with diseases
using traditional statistical models. For example it is not possible to
classify individuals as having healthy colons or screen relevant
neoplasia using Bray-Curtis distances based on the 16S rRNA gene
sequences collected from fecal samples {[}Figure S1{]}. This variation
is likely due to the ability of many bacterial populations to fill the
same niche such that different populations cause the same disease in
different individuals. Furthermore, a growing number of studies have
shown that it is rare for a single bacterial species to be associated
with a disease. Instead, subsets of the microbiome account for
differences in health. Traditional statistical approaches do not
adequately account for the variation in the human microbiome and
typically consider the protective or risk effects of each bacterial
population separately (1). Recently, machine learning (ML) models have
grown in popularity among microbiome researchers as our ability to
sample large numbers of individuals has grown; such models can
effectively account for the interpersonal microbiome variation and the
ecology of disease because they consider the relative abundance of each
bacterial population in the context of others rather than in isolation.

ML models can be used to increase our understanding of the variation in
the structure of existing data and in making predictions about new data.
Researchers have used ML models to diagnose and understand the
ecological basis of diseases such as liver cirrhosis, colorectal cancer,
inflammatory bowel diseases, obesity, and type 2 diabetes (2--19). The
task of diagnosing an individual relies on a rigorously validated model.
However, there are common methodological and reporting problems that
arise when applying ML to such data that need to be addressed for the
field to progress. These problems include a lack of transparency in
which methods are used and how these methods are implemented; evaluating
models without separate held-out test data; unreported variation between
the predictive performance on different folds of cross-validation; and
unreported variation between cross-validation and testing performances.
Though the microbiome field is making progress to avoid some of these
pitfalls including validating their models on independent datasets (8,
19, 20) and introducing accessible and open-source ML tools (21--24),
more work is needed to improve reproducibility further and minimize
overestimating for model performance.

Among microbiome researchers, the lack of justification when selecting a
modeling approach has often been due to an implicit assumption that more
complex models are better. This has resulted in a trend towards using
non-linear models such as random forest and deep neural networks (3, 12,
25--27) over simpler models such as logistic regression or other linear
models (19, 23, 28). Although in some cases, complex models may capture
important non-linear relationships and therefore yield better
predictions, they can also result in black boxes that lack
interpretability. Such models require post hoc explanations to quantify
the importance of each feature in making predictions. Depending on the
goal of the modeling, other approaches may be more appropriate. For
example, researchers trying to identify the microbiota associated with
disease may desire a more interpretable model, whereas clinicians may
emphasize predictive performance. Nonetheless, it is essential to
understand that the benefit of more complex, less interpretable models
may be minimal (29--31). It is important for researchers to justify
their choice of modeling approach.

In this study, we provided steps toward standardization of machine
learning methods for microbiome studies which are often poorly
documented and executed. To showcase a rigorous ML pipeline and to shed
light on how ML model selection can affect modeling results, we
performed an empirical analysis comparing the predictive performance,
interpretability, data requirements, and training times of seven
modeling approaches with the same dataset and pipeline. We built three
linear models with different forms of regularization: L2-regularized
logistic regression and L1 and L2-regularized support vector machines
(SVM) with a linear kernel. We also trained four non-linear models: SVM
with radial basis function kernel, a decision tree, random forest, and
gradient boosted trees. We compared their predictive performance,
interpretability, and training time. To demonstrate the performance of
these modeling approaches and our pipeline, we present a case study
using data from a previously published study that sought to classify
individuals as having healthy colons or colonic lesions based on the 16S
rRNA gene sequences collected from fecal samples (4). This dataset was
selected because it is a relatively large collection of individuals
(N=490) connected to a clinically significant disease where there is
ample evidence that the disease is driven by variation in the microbiome
(2, 4, 5, 32). With this dataset, we developed an ML pipeline that can
be used in many different scenarios for training and evaluating models.
This framework can be easily applied to other host-associated and
environmental microbiome datasets. We also provided an aspirational
rubric for evaluating the rigor of ML practices applied to microbiome
data {[}Table S1{]} to urge the audience to be diligent in their study
design and model selection, development, evaluation, and interpretation.

\hypertarget{results}{%
\subsection{Results}\label{results}}

\textbf{Model selection and pipeline construction}. We established a
reusable ML pipeline for model selection and evaluation, focusing on
seven different commonly used supervised learning algorithms {[}Figure
1, Table 1{]}.

First, we randomly split the data into training and test sets so that
the training set consisted of 80\% of the full dataset, while the test
set was composed of the remaining 20\% {[}Figure 1{]}. To maintain the
distribution of controls and cases found in the full dataset, we
performed stratified splits. For example, our full dataset included 490
individuals. Of these, 261 had healthy colons (53\%) and 229 had a
screen relevant neoplasia (SRN; 46.7\%). A training set included 393
individuals, of which 209 had an SRN (53\%), while the test set was
composed of 97 individuals, of which 52 had an SRN (54\%). The training
data were used to build and select the models, and the test set was used
for evaluating the model. We trained seven different models using the
training data {[}Table 1{]}.

Model selection requires tuning hyperparameters. Hyperparameters are
parameters that need to be specified or tuned by the user, in order to
train a model for a specific modeling problem. For example, when using
regularization, C is a hyperparameter that indicates the penalty for
overfitting. Hyperparameters are tuned using the training data to find
the best model. We selected hyperparameters by performing repeated
five-fold cross-validation (CV) on the training set {[}Figure 1{]}. The
five-fold CV was also stratified to maintain the overall case and
control distribution. We chose the hyperparameter values that led to the
best average CV predictive performance using the area under the receiver
operating characteristic curve (AUROC) {[}Figure S2 and S3{]}. The AUROC
ranges from 0, where the model's predictions are perfectly incorrect, to
1.0, where the model perfectly distinguishes between cases and controls.
An AUROC value of 0.5 indicates that the model's predictions are no
different than random. To select hyperparameters, we performed a grid
search for hyperparameter settings when training the models. Default
hyperparameter settings in developed ML packages available in R, Python,
and MATLAB programming languages may be inadequate for effective
application of classification algorithms and need to be optimized for
each new ML task. For example, L1-regularized SVM with linear kernel
showed large variability between different regularization strengths (C)
and benefited from tuning as the default C parameter was 1 {[}Figure
S2{]}.

Once hyperparameters were selected, we trained the model using the full
training dataset and applied the final model to the held-out data to
evaluate the testing predictive performance of each model. The
data-split, hyperparameter selection, training and testing steps were
repeated 100 times to obtain a robust interpretation of model
performance, less likely to be affected by a ``lucky'' or ``unlucky''
split {[}Figure 1{]}.

\textbf{Predictive performance and generalizability of the seven
models.} We evaluated the predictive performance of the seven models to
classify individuals as having healthy colons or SRNs {[}Figure 2{]}.
The predictive performance of random forest model was higher than other
ML models with a median 0.695 {[}IQR 0.650-0.739{]}, though not
significantly (p=0.5; The p-value was manually calculated using the
sampling distribution of the test statistic under the null hypothesis)
(Figure S4). Similarly, L2-regularized logistic regression, XGBoost,
L2-regularized SVM with linear and radial basis function kernel AUROC
values were not significantly different from one another and had median
AUROC values of 0.680 {[}IQR 0.639-0.750{]}, 0.679 {[}IQR
0.643-0.746{]}, 0.678 {[}IQR 0.639-0.750{]} and 0.668 {[}IQR
0.639-0.750{]}, respectively. L1-regularized SVM with linear kernel and
decision tree had significantly lower AUROC values than the other ML
models with median AUROC of 0.650 {[}IQR 0.629-0.760{]} and 0.601 {[}IQR
0.636-0.753{]}, respectively {[}Figure 2{]}. Interestingly, these
results demonstrate that the most complex model (XGBoost) did not have
the best performance and that the most interpretable models
(L2-regularized logistic regression and L2-regularized SVM with linear
kernel) performed nearly as well as non-linear models.

To evaluate the generalizability of each model, we compared the median
cross-validation AUROC to the median testing AUROC. If the difference
between the cross-validation and testing AUROCs was large, then that
could indicate that the models were overfit to the training data. The
largest difference in median AUROCs was 0.021 in L1-regularized SVM with
linear kernel, followed by SVM with radial basis function kernel and
decision tree with a difference of 0.007 and 0.006, respectively
{[}Figure 2{]}. These differences were relatively small and gave us
confidence in our estimate of the generalization performance of the
models.

To evaluate the variation in the estimated performance, we calculated
the range of AUROC values for each model using 100 data-splits. The
range among the testing AUROC values within each model varied by 0.230
on average across the seven models. If we had only done a single split,
then there is a risk that we could have gotten lucky or unlucky in
estimating model performance. For instance, the lowest AUROC value of
the random forest model was 0.593, whereas the highest was 0.810. These
results showed that depending on the data-split, the testing performance
can vary {[}Figure 2{]}. Therefore, it is important to employ multiple
data splits when estimating generalization performance.

To show the effect of sample size on model generalizability, we compared
cross-validation AUROC values of L2-regularized logistic regression and
random forest models when we subsetted our original study design with
490 subjects to 15, 30, 60, 120, and 245 subjects {[}Figure S5{]}. The
variation in cross-validation performance within both models at smaller
sample sizes was larger than when the full collection of samples was
used to train and validate the models. Because of the high
dimensionality of the microbiome data (6920 OTUs), large sample sizes
can lead to better models.

\textbf{Interpretation of each ML model.} We often use ML models not
just to predict a health outcome, but also to identify potential
biomarkers for disease. Therefore, model interpretation becomes crucial
for microbiome studies. Interpretability is related to the degree to
which humans can understand the reasons behind a model prediction
(33--35). ML models often decrease in interpretability as they increase
in complexity. In this study, we used two methods to help interpret our
models.

First, we interpreted the feature importance of the linear models (L1
and L2-regularized SVM with linear kernel and L2-regularized logistic
regression) using the median rank of absolute feature weights for each
OTU {[}Figure 3{]}. We also reviewed the signs of feature weights to
determine whether an OTU was associated with classifying a subject as
being healthy or having an SRN. It was encouraging that many of the
highest-ranked OTUs were shared across these three models (e.g., OTUs
50, 426, 609, 822, 1239). The benefit of this approach was knowing the
sign and magnitude of each OTU coefficient in the trained model. This
allowed us to immidiately interpret negative and positive coefficient
signs as protective and risk factors, respectively and the magnitude as
the impact of these factors. However, this approach is limited to linear
models or models with prespecified interaction terms.

Second, to analyze non-linear models, we interpreted the feature
importance using permutation importance (36). Whereas the absolute
feature weights were determined from the trained models, here we
measured importance using the held-out test data. Permutation importance
analysis is a post hoc explanation of the model, in which we randomly
permuted groups of perfectly correlated features together and other
features individually across the two groups in the held-out test data
{[}Figure S6{]}. We then calculated how much the predictive performance
of the model (i.e., testing AUROC values) decreased when each OTU or
group of OTUs was randomly permuted. We ranked the OTUs based on how
much the median testing AUROC decreased when it was permuted; the OTU
with the largest decrease ranked highest {[}Figure 4{]}. Among the
twenty OTUs with the largest impact, there was only one OTU (OTU 822)
that was shared among all of the models; however, we found three OTUs
(OTUs 58, 110, 367) that were important in each of the tree-based
models. Similarly, the random forest and XGBoost models shared four of
the most important OTUs (OTUs 2, 12, 361, 477). Permutation analysis
results also revealed that with the exception of the decision tree
model, removal of any individual OTU had minimal impact on model
performance. For example, if OTU 367 was permuted across the samples in
the decision tree model, the median AUROC dropped from 0.601 to 0.525.
In contrast, if the same OTU was permuted in the random forest model,
the AUROC only dropped from 0.695 to 0.680, which indicated high degree
of collinearity in the dataset. Permutation analysis allowed us to gauge
the importance of each OTU in non-linear models and partially account
for collinearity by grouping correlated OTUs to determine their impact
as a group.

To further highlight the differences between the two interpretation
methods, we used permutation importance to interpret the linear models
{[}Figure S7{]}. When we analyzed the L1-regularized SVM with linear
kernel model using feature rankings based on weights {[}Figure 3{]} and
permutation importance {[}Figure S7{]}, 17 of the 20 top OTUs (e.g., OTU
609, 822, 1239) were deemed important by both interpretation methods.
Similarly, for the L2-regularized SVM and L2-regularized logistic
regression, 9 and 12 OTUs, respectively, were shared among the two
interpretation methods. These results indicate that both methods are
consistent in selecting the most important OTUs.

We also compared the top 20 OTUs selected by permutation importance in
L2-regularized logistic regression {[}Figure S7{]} and the highest
performing tree-based models, random forest and XGBoost {[}Figure 4{]}.
Two and five OTUs, respectively, were shared among the models. These
results indicate that we were able to identify important OTUs that are
shared across the highest performing linear and non-linear models when
we use permutation importance as our interpretation method.

We then evaluated the difference in relative abundances of the top 20
OTUs identified in L2-regularized logistic regression and random forest
models between healthy patients and patients with SRNs {[}Figure S8{]}.
There were minimal differences in the median relative abundances across
OTUs between different diagnoses. This supports our claim that it is not
possible to differentiate disease versus healthy states by focusing on
individual taxa. The ability for ML models to simultaneously consider
the relative abundances of multiple OTUs and their context dependency is
a great advantage over traditional statistical approaches that consider
each OTU in isolation.

\textbf{The computational efficiency of each ML model.} We compared the
training times of the seven ML models. The training times increased with
the complexity of the model and the number of potential hyperparameter
combinations. Also, the linear models trained faster than non-linear
models {[}Figures S2-S3; Figure 5{]}.

\hypertarget{discussion}{%
\subsection{Discussion}\label{discussion}}

There is a growing awareness that many human diseases and environmental
processes are not driven by a single organism but are the product of
multiple bacterial populations. Traditional statistical approaches are
useful for identifying those cases where a single organism is associated
with a process. In contrast, ML methods offer the ability to incorporate
the structure of the microbial communities as a whole and identify
associations between community structure and disease state. If it is
possible to classify communities reliably, then ML methods also offer
the ability to identify those microbial populations within the
communities that are responsible for the classification. However, the
application of ML in microbiome studies is still in its infancy, and the
field needs to develop a better understanding of different ML methods,
their strengths and weaknesses, and how to implement them.

To address these needs, we developed an open-sourced framework for ML
models. Using this pipeline, we benchmarked seven ML models and showed
that the tradeoff between model complexity and performance may be less
severe than originally hypothesized. In terms of predictive performance,
the random forest model had the best AUROC compared to the other six
models. However, the second-best model was L2-regularized logistic
regression with a median AUROC difference of less than 0.015 compared to
random forest. While our implementation of random forest took 83.2 hours
to train, our L2-regularized logistic regression trained in 12 minutes.
In terms of interpretability, random forest is a non-linear ML model,
while L2-regularized logistic regression, a linear model, was more
easily interpreted because we could use the feature weights. Comparing
many different models showed us that the most complex model was not
necessarily the best model for our ML task.

We established a pipeline that can be generalized to any modeling method
that predicts a binary health outcome. We performed a random data-split
to create a training set (80\% of the data) and a held-out test set
(20\% of the data), which we used to evaluate predictive performance. We
used the AUROC metric to evaluate predictive performance as it is a
clinically relevant evaluation metric for our study. We repeated this
data-split 100 times to measure the possible variation in predictive
performance. During training, we tuned the model hyperparameters with a
repeated five-fold cross-validation. Despite the high number of features
microbiome datasets typically have, the models we built with this
pipeline generalized to the held-out test sets.

We highlighted the importance of model interpretation to gain greater
biological insights into microbiota-associated diseases. In this study,
we showcased two different interpretation methods: ranking each OTU by
(i) their absolute weights in the trained models and (ii) their impact
on the predictive performance based on permutation importance. Previous
studies have emphasized the difficulty of interpreting the feature
coefficients in linear models (37) and the biases introduced by
computing feature importance using built-in methods (e.g., gini drop) of
tree-based models (38). Therefore, we encourage our audience to use both
interpretation methods highlighted in this study as permutation
importance is a model-agnostic tool that can be used to compared feature
importance across different models. Human-associated microbial
communities have complex correlation structures that create collinearity
in the datasets. This can hinder our ability to reliably interpret
models because the feature weights of correlated OTUs are influenced by
one another (39). To capture all important features, once we identify
highly ranked OTUs, we should review their relationships with other
OTUs. These relationships will help us generate new hypotheses about the
ecology of the disease and test them with follow-up experiments. When we
used permutation importance, we partially accounted for collinearity by
grouping correlated OTUs to determine their impact as a group. We
grouped OTUs that had a perfect correlation with each other; however, we
could reduce the correlation threshold to further investigate the
relationships among correlated features. By our approach, we identified
432 OTUs out of 6,920 that had perfect correlations with at least one
other OTU. The decision to establish correlation thresholds is left to
researchers to implement for their own analyses. Regardless of the
threshold, undestanding the correlation structures within the data is
critical to avoid misinterpreting the models. Such structures are likely
to be a particular problem with shotgun metagenomic datasets where
collinearity will be more pronounced due to many genes being correlated
with one another because they come from the same chromosome. Finally,
true causal mechanisms (e.g., role of microbiome in colorectal cancer)
cannot be explained solely by the highest performing machine learning
model (40). To identify the true underlying microbial factors of a
disease, it is crucial to follow up on any correlation analyses with
further hypothesis testing and experimentation for biological
validation.

In this study, we did not consider all possible modeling approaches.
However, the principles highlighted throughout this study apply to other
ML modeling tasks with microbiome data. For example, we did not evaluate
multicategory classification methods to predict non-binary outcomes. We
could have trained models to differentiate between people with healthy
colons and those with adenomas or carcinomas (k=3 categories). We did
not perform this analysis because the clinically relevant diagnosis
grouping was between patients with healthy colons and those with SRNs.
Furthermore, as the number of classes increases, more samples are
required for each category to train an accurate model. We also did not
use regression-based analyses to predict a non-categorical outcome. We
have previously used such an approach to train random forest models to
predict fecal short-chain fatty acid concentrations based on microbiome
data (41). Our analysis was also limited to shallow learning methods and
did not explore deep learning methods such as neural networks. Deep
learning methods hold promise (12, 42, 43), but microbiome datasets
often suffer from having many features and small sample sizes, which
result in overfitting.

Our framework provides a reproducible structure for investigators
wanting to train, evaluate, and interpret their own ML models to
generate hypotheses regarding which OTUs might be biologically relevant.
However, deploying microbiome-based models to make clinical diagnoses or
predictions is a significantly more challenging and distinct undertaking
(44). For example, we currently lack standardized methods to collect
patient samples, generate sequence data, and report clinical data. We
are also challenged by the practical constraints of OTU-based
approaches. The de novo algorithms commonly in use are slow, require
considerable memory, and result in different OTU assignments as new data
are added (45). Finally, we also need independent validation cohorts to
test the performance of a diagnostic model. To realize the potential for
using ML approaches with microbiome data, it is necessary that we direct
our efforts to overcome these challenges.

Our study highlights the need to make educated choices at every step of
developing an ML model with microbiome data. We created an aspirational
rubric that researchers can use to identify potential pitfalls when
using ML in microbiome studies and ways to avoid them {[}Table S1{]}. We
highlighted the trade-offs between model complexity and
interpretability, the need for tuning hyperparameters, the utility of
held-out test sets for evaluating predictive performance, and the
importance of considering correlation structures in datasets for
reliable interpretation. We showed the importance of interpretability
for generating hypotheses to identify causal, biological relationships
and for identifying inconsistencies in model setup. Furthermore, we
underscored the importance of proper experimental design and methods to
help us achieve the level of validity and accountability we want from
models built for patient health.

\hypertarget{materials-and-methods}{%
\subsection{Materials and Methods}\label{materials-and-methods}}

\textbf{Data collection and study population.} The original stool
samples described in our analysis were obtained from patients recruited
by Great Lakes-New England Early Detection Research Network (5). Stool
samples were provided by adults who were undergoing a scheduled
screening or surveillance colonoscopy. Participants were recruited from
Toronto (ON, Canada), Boston (MA, USA), Houston (TX, USA), and Ann Arbor
(MI, USA). Patients' colonic health was visually assessed by colonoscopy
with bowel preparation and tissue histopathology of all resected
lesions. We assigned patients into two classes: those with healthy
colons and those with screen relevant neoplasias (SRNs). The healthy
class included patients with healthy colons or non-advanced adenomas,
whereas the SRN class included patients with advanced adenomas or
carcinomas (46). Patients with an adenoma greater than 1 cm, more than
three adenomas of any size, or an adenoma with villous histology were
classified as having advanced adenomas (46). There were 172 patients
with normal colonoscopies, 198 with adenomas, and 120 with carcinomas.
Of the 198 adenomas, 109 were identified as advanced adenomas. Together
261 patients were classified as healthy and 229 patients were classified
as having an SRN.

\textbf{16S rRNA gene sequencing data.} Stool samples provided by the
patients were used for 16S rRNA gene sequencing to measure bacterial
population abundances. The sequence data used in our analyses were
originally generated by Baxter et al.~(available through NCBI Sequence
Read Archive {[}SRP062005{]}, (5)). The OTU abundance table was
generated by Sze et al (47), who processed the 16S rRNA sequences in
mothur (v1.39.3) using the default quality filtering methods,
identifying and removing chimeric sequences using VSEARCH, and assigning
to OTUs at 97\% similarity using the OptiClust algorithm (45, 48, 49);
(\url{https://github.com/SchlossLab/Sze_CRCMetaAnalysis_mBio_2018/blob/master/data/process/baxter/baxter.0.03.subsample.shared}).
These OTU abundances were the features we used to predict colorectal
health of the patients. There were 6920 OTUs. OTU abundances were
subsampled to the size of the smallest sample and normalized across
samples such that the highest abundance of each OTU would be 1, and the
lowest would be 0.

\textbf{Model training and evaluation.} Models were trained using the
caret package (v.6.0.81) in R (v.3.5.0). We modified the caret code to
calculate decision values for models generated using L2-regularized SVM
with linear kernel and L1-regularized SVM with linear kernel. The code
for these changes on L2-regularized SVM with linear kernel and
L1-regularized SVM with linear kernel models are available at
\url{https://github.com/SchlossLab/Topcuoglu_ML_XXX_2019/blob/master/data/caret_models/svmLinear3.R}
and at
\url{https://github.com/SchlossLab/Topcuoglu_ML_XXX_2019/blob/master/data/caret_models/svmLinear4.R},
respectively.

For hyperparameter selection, we started with a granular grid search.
Then we narrowed and fine-tuned the range of each hyperparameter.For
L2-regularized logistic regression, L1 and L2-regularized SVM with
linear and radial basis function kernels, we tuned the cost
hyperparameter, which controls the regularization strength, where
smaller values specify stronger regularization. For SVM with radial
basis function kernel, we also tuned the sigma hyperparameter, which
determines the reach of a single training instance where, for a high
value of sigma, the SVM decision boundary will be dependent on the
points that are closest to the decision boundary. For the decision tree
model, we tuned the depth of the tree where the deeper the tree, the
more splits it has. For random forest, we tuned the number of features
to consider when looking for the best tree split. For XGBoost, we tuned
the learning rate and the fraction of samples used for fitting the
individual base learners. Performing a grid search for hyperparameter
selection might not be feasible for when there are more than two
hyperparameters to tune for. In such cases, it is more efficient to use
random search or recently developed tools such as Hyperband to identify
good hyperparameter configurations (50).

The computational burden during model training due to model complexity
was reduced by parallelizing segments of the ML pipeline. We
parallelized the training of each data-split. This allowed the 100
data-splits to be processed through the ML pipeline simultaneously at
the same time for each model. It is possible to further parallelize the
cross-validation step for each hyperparameter setting which we have not
performed in this study.

\textbf{Permutation importance workflow.} We calculated a Spearman's
rank-order correlation matrix and defined correlated OTUs as having
perfect correlation (correlation coefficient = 1 and p \textless{}
0.01). OTUs without a perfect correlation to each other were permuted
individually, whereas correlated ones were grouped together and permuted
at the same time.

\textbf{Statistical analysis workflow.} Data summaries, statistical
analysis, and data visualizations were performed using R (v.3.5.0) with
the tidyverse package (v.1.2.1). We compared the performance of the
models pairwise by calculating the difference between AUROC values from
the same data-split (for 100 data-splits). We determined if the models
were significantly different by calculating the empirical p-value (2 x
min(\% of AUROC differences \(\geq\) 0, \% of AUROC differences \(\leq\)
0) for the double tail event (e.g., Figure S4).

\textbf{Code availability.} The code for all sequence curation and
analysis steps including an Rmarkdown version of this manuscript is
available at \url{https://github.com/SchlossLab/Topcuoglu_ML_XXX_2019/}.

\textbf{Acknowledgements.} We thank all the study participants of Great
Lakes-New England Early Detection Research Network. We would like to
thank the members of the Schloss lab for their valuable feedback. Salary
support for MR came from NIH grant 1R01CA215574. Salary support for PDS
came from NIH grants P30DK034933 and 1R01CA215574.

\newpage

\hypertarget{references}{%
\subsection{References}\label{references}}

\hypertarget{refs}{}
\leavevmode\hypertarget{ref-segata_metagenomic_2011}{}%
1. \textbf{Segata N}, \textbf{Izard J}, \textbf{Waldron L},
\textbf{Gevers D}, \textbf{Miropolsky L}, \textbf{Garrett WS},
\textbf{Huttenhower C}. 2011. Metagenomic biomarker discovery and
explanation. Genome Biol \textbf{12}:R60.
doi:\href{https://doi.org/10.1186/gb-2011-12-6-r60}{10.1186/gb-2011-12-6-r60}.

\leavevmode\hypertarget{ref-zeller_potential_2014}{}%
2. \textbf{Zeller G}, \textbf{Tap J}, \textbf{Voigt AY},
\textbf{Sunagawa S}, \textbf{Kultima JR}, \textbf{Costea PI},
\textbf{Amiot A}, \textbf{Böhm J}, \textbf{Brunetti F},
\textbf{Habermann N}, \textbf{Hercog R}, \textbf{Koch M},
\textbf{Luciani A}, \textbf{Mende DR}, \textbf{Schneider MA},
\textbf{Schrotz-King P}, \textbf{Tournigand C}, \textbf{Tran Van Nhieu
J}, \textbf{Yamada T}, \textbf{Zimmermann J}, \textbf{Benes V},
\textbf{Kloor M}, \textbf{Ulrich CM}, \textbf{Knebel Doeberitz M von},
\textbf{Sobhani I}, \textbf{Bork P}. 2014. Potential of fecal microbiota
for early-stage detection of colorectal cancer. Mol Syst Biol
\textbf{10}.
doi:\href{https://doi.org/10.15252/msb.20145645}{10.15252/msb.20145645}.

\leavevmode\hypertarget{ref-zackular_human_2014}{}%
3. \textbf{Zackular JP}, \textbf{Rogers MAM}, \textbf{Ruffin MT},
\textbf{Schloss PD}. 2014. The human gut microbiome as a screening tool
for colorectal cancer. Cancer Prev Res \textbf{7}:1112--1121.
doi:\href{https://doi.org/10.1158/1940-6207.CAPR-14-0129}{10.1158/1940-6207.CAPR-14-0129}.

\leavevmode\hypertarget{ref-baxter_dna_2016}{}%
4. \textbf{Baxter NT}, \textbf{Koumpouras CC}, \textbf{Rogers MAM},
\textbf{Ruffin MT}, \textbf{Schloss PD}. 2016. DNA from fecal
immunochemical test can replace stool for detection of colonic lesions
using a microbiota-based model. Microbiome \textbf{4}.
doi:\href{https://doi.org/10.1186/s40168-016-0205-y}{10.1186/s40168-016-0205-y}.

\leavevmode\hypertarget{ref-baxter_microbiota-based_2016}{}%
5. \textbf{Baxter NT}, \textbf{Ruffin MT}, \textbf{Rogers MAM},
\textbf{Schloss PD}. 2016. Microbiota-based model improves the
sensitivity of fecal immunochemical test for detecting colonic lesions.
Genome Medicine \textbf{8}:37.
doi:\href{https://doi.org/10.1186/s13073-016-0290-3}{10.1186/s13073-016-0290-3}.

\leavevmode\hypertarget{ref-hale_shifts_2017}{}%
6. \textbf{Hale VL}, \textbf{Chen J}, \textbf{Johnson S},
\textbf{Harrington SC}, \textbf{Yab TC}, \textbf{Smyrk TC},
\textbf{Nelson H}, \textbf{Boardman LA}, \textbf{Druliner BR},
\textbf{Levin TR}, \textbf{Rex DK}, \textbf{Ahnen DJ}, \textbf{Lance P},
\textbf{Ahlquist DA}, \textbf{Chia N}. 2017. Shifts in the fecal
microbiota associated with adenomatous polyps. Cancer Epidemiol
Biomarkers Prev \textbf{26}:85--94.
doi:\href{https://doi.org/10.1158/1055-9965.EPI-16-0337}{10.1158/1055-9965.EPI-16-0337}.

\leavevmode\hypertarget{ref-pasolli_machine_2016}{}%
7. \textbf{Pasolli E}, \textbf{Truong DT}, \textbf{Malik F},
\textbf{Waldron L}, \textbf{Segata N}. 2016. Machine learning
meta-analysis of large metagenomic datasets: Tools and biological
insights. PLoS Comput Biol \textbf{12}.
doi:\href{https://doi.org/10.1371/journal.pcbi.1004977}{10.1371/journal.pcbi.1004977}.

\leavevmode\hypertarget{ref-sze_looking_2016}{}%
8. \textbf{Sze MA}, \textbf{Schloss PD}. 2016. Looking for a signal in
the noise: Revisiting obesity and the microbiome. mBio \textbf{7}.
doi:\href{https://doi.org/10.1128/mBio.01018-16}{10.1128/mBio.01018-16}.

\leavevmode\hypertarget{ref-walters_meta-analyses_2014}{}%
9. \textbf{Walters WA}, \textbf{Xu Z}, \textbf{Knight R}. 2014.
Meta-analyses of human gut microbes associated with obesity and IBD.
FEBS Lett \textbf{588}:4223--4233.
doi:\href{https://doi.org/10.1016/j.febslet.2014.09.039}{10.1016/j.febslet.2014.09.039}.

\leavevmode\hypertarget{ref-vazquez-baeza_guiding_2018}{}%
10. \textbf{Vázquez-Baeza Y}, \textbf{Gonzalez A}, \textbf{Xu ZZ},
\textbf{Washburne A}, \textbf{Herfarth HH}, \textbf{Sartor RB},
\textbf{Knight R}. 2018. Guiding longitudinal sampling in IBD cohorts.
Gut \textbf{67}:1743--1745.
doi:\href{https://doi.org/10.1136/gutjnl-2017-315352}{10.1136/gutjnl-2017-315352}.

\leavevmode\hypertarget{ref-qin_alterations_2014}{}%
11. \textbf{Qin N}, \textbf{Yang F}, \textbf{Li A}, \textbf{Prifti E},
\textbf{Chen Y}, \textbf{Shao L}, \textbf{Guo J}, \textbf{Le Chatelier
E}, \textbf{Yao J}, \textbf{Wu L}, \textbf{Zhou J}, \textbf{Ni S},
\textbf{Liu L}, \textbf{Pons N}, \textbf{Batto JM}, \textbf{Kennedy SP},
\textbf{Leonard P}, \textbf{Yuan C}, \textbf{Ding W}, \textbf{Chen Y},
\textbf{Hu X}, \textbf{Zheng B}, \textbf{Qian G}, \textbf{Xu W},
\textbf{Ehrlich SD}, \textbf{Zheng S}, \textbf{Li L}. 2014. Alterations
of the human gut microbiome in liver cirrhosis. Nature
\textbf{513}:59--64.
doi:\href{https://doi.org/10.1038/nature13568}{10.1038/nature13568}.

\leavevmode\hypertarget{ref-geman_deep_2018}{}%
12. \textbf{Geman O}, \textbf{Chiuchisan I}, \textbf{Covasa M},
\textbf{Doloc C}, \textbf{Milici M-R}, \textbf{Milici L-D}. 2018. Deep
learning tools for human microbiome big data, pp. 265--275. \emph{In}
Balas, VE, Jain, LC, Balas, MM (eds.), Soft computing applications.
Springer International Publishing.

\leavevmode\hypertarget{ref-thaiss_persistent_2016}{}%
13. \textbf{Thaiss CA}, \textbf{Itav S}, \textbf{Rothschild D},
\textbf{Meijer MT}, \textbf{Levy M}, \textbf{Moresi C},
\textbf{Dohnalová L}, \textbf{Braverman S}, \textbf{Rozin S},
\textbf{Malitsky S}, \textbf{Dori-Bachash M}, \textbf{Kuperman Y},
\textbf{Biton I}, \textbf{Gertler A}, \textbf{Harmelin A},
\textbf{Shapiro H}, \textbf{Halpern Z}, \textbf{Aharoni A},
\textbf{Segal E}, \textbf{Elinav E}. 2016. Persistent microbiome
alterations modulate the rate of post-dieting weight regain. Nature
\textbf{540}:544--551.
doi:\href{https://doi.org/10.1038/nature20796}{10.1038/nature20796}.

\leavevmode\hypertarget{ref-dadkhah_gut_2019}{}%
14. \textbf{Dadkhah E}, \textbf{Sikaroodi M}, \textbf{Korman L},
\textbf{Hardi R}, \textbf{Baybick J}, \textbf{Hanzel D}, \textbf{Kuehn
G}, \textbf{Kuehn T}, \textbf{Gillevet PM}. 2019. Gut microbiome
identifies risk for colorectal polyps. BMJ Open Gastroenterology
\textbf{6}:e000297.
doi:\href{https://doi.org/10.1136/bmjgast-2019-000297}{10.1136/bmjgast-2019-000297}.

\leavevmode\hypertarget{ref-flemer_oral_2018}{}%
15. \textbf{Flemer B}, \textbf{Warren RD}, \textbf{Barrett MP},
\textbf{Cisek K}, \textbf{Das A}, \textbf{Jeffery IB}, \textbf{Hurley
E}, \textbf{O`Riordain M}, \textbf{Shanahan F}, \textbf{O`Toole PW}.
2018. The oral microbiota in colorectal cancer is distinctive and
predictive. Gut \textbf{67}:1454--1463.
doi:\href{https://doi.org/10.1136/gutjnl-2017-314814}{10.1136/gutjnl-2017-314814}.

\leavevmode\hypertarget{ref-montassier_pretreatment_2016}{}%
16. \textbf{Montassier E}, \textbf{Al-Ghalith GA}, \textbf{Ward T},
\textbf{Corvec S}, \textbf{Gastinne T}, \textbf{Potel G}, \textbf{Moreau
P}, \textbf{Cochetiere MF de la}, \textbf{Batard E}, \textbf{Knights D}.
2016. Pretreatment gut microbiome predicts chemotherapy-related
bloodstream infection. Genome Medicine \textbf{8}:49.
doi:\href{https://doi.org/10.1186/s13073-016-0301-4}{10.1186/s13073-016-0301-4}.

\leavevmode\hypertarget{ref-ai_systematic_2017}{}%
17. \textbf{Ai L}, \textbf{Tian H}, \textbf{Chen Z}, \textbf{Chen H},
\textbf{Xu J}, \textbf{Fang J-Y}. 2017. Systematic evaluation of
supervised classifiers for fecal microbiota-based prediction of
colorectal cancer. Oncotarget \textbf{8}:9546--9556.
doi:\href{https://doi.org/10.18632/oncotarget.14488}{10.18632/oncotarget.14488}.

\leavevmode\hypertarget{ref-dai_multi-cohort_2018}{}%
18. \textbf{Dai Z}, \textbf{Coker OO}, \textbf{Nakatsu G}, \textbf{Wu
WKK}, \textbf{Zhao L}, \textbf{Chen Z}, \textbf{Chan FKL},
\textbf{Kristiansen K}, \textbf{Sung JJY}, \textbf{Wong SH}, \textbf{Yu
J}. 2018. Multi-cohort analysis of colorectal cancer metagenome
identified altered bacteria across populations and universal bacterial
markers. Microbiome \textbf{6}:70.
doi:\href{https://doi.org/10.1186/s40168-018-0451-2}{10.1186/s40168-018-0451-2}.

\leavevmode\hypertarget{ref-mossotto_classification_2017}{}%
19. \textbf{Mossotto E}, \textbf{Ashton JJ}, \textbf{Coelho T},
\textbf{Beattie RM}, \textbf{MacArthur BD}, \textbf{Ennis S}. 2017.
Classification of paediatric inflammatory bowel disease using machine
learning. Scientific Reports \textbf{7}.
doi:\href{https://doi.org/10.1038/s41598-017-02606-2}{10.1038/s41598-017-02606-2}.

\leavevmode\hypertarget{ref-wong_quantitation_2017}{}%
20. \textbf{Wong SH}, \textbf{Kwong TNY}, \textbf{Chow T-C}, \textbf{Luk
AKC}, \textbf{Dai RZW}, \textbf{Nakatsu G}, \textbf{Lam TYT},
\textbf{Zhang L}, \textbf{Wu JCY}, \textbf{Chan FKL}, \textbf{Ng SSM},
\textbf{Wong MCS}, \textbf{Ng SC}, \textbf{Wu WKK}, \textbf{Yu J},
\textbf{Sung JJY}. 2017. Quantitation of faecal fusobacterium improves
faecal immunochemical test in detecting advanced colorectal neoplasia.
Gut \textbf{66}:1441--1448.
doi:\href{https://doi.org/10.1136/gutjnl-2016-312766}{10.1136/gutjnl-2016-312766}.

\leavevmode\hypertarget{ref-statnikov_comprehensive_2013}{}%
21. \textbf{Statnikov A}, \textbf{Henaff M}, \textbf{Narendra V},
\textbf{Konganti K}, \textbf{Li Z}, \textbf{Yang L}, \textbf{Pei Z},
\textbf{Blaser MJ}, \textbf{Aliferis CF}, \textbf{Alekseyenko AV}. 2013.
A comprehensive evaluation of multicategory classification methods for
microbiomic data. Microbiome \textbf{1}:11.
doi:\href{https://doi.org/10.1186/2049-2618-1-11}{10.1186/2049-2618-1-11}.

\leavevmode\hypertarget{ref-knights_supervised_2011}{}%
22. \textbf{Knights D}, \textbf{Costello EK}, \textbf{Knight R}. 2011.
Supervised classification of human microbiota. FEMS Microbiology Reviews
\textbf{35}:343--359.
doi:\href{https://doi.org/10.1111/j.1574-6976.2010.00251.x}{10.1111/j.1574-6976.2010.00251.x}.

\leavevmode\hypertarget{ref-wirbel_meta-analysis_2019}{}%
23. \textbf{Wirbel J}, \textbf{Pyl PT}, \textbf{Kartal E}, \textbf{Zych
K}, \textbf{Kashani A}, \textbf{Milanese A}, \textbf{Fleck JS},
\textbf{Voigt AY}, \textbf{Palleja A}, \textbf{Ponnudurai R},
\textbf{Sunagawa S}, \textbf{Coelho LP}, \textbf{Schrotz-King P},
\textbf{Vogtmann E}, \textbf{Habermann N}, \textbf{Niméus E},
\textbf{Thomas AM}, \textbf{Manghi P}, \textbf{Gandini S},
\textbf{Serrano D}, \textbf{Mizutani S}, \textbf{Shiroma H},
\textbf{Shiba S}, \textbf{Shibata T}, \textbf{Yachida S}, \textbf{Yamada
T}, \textbf{Waldron L}, \textbf{Naccarati A}, \textbf{Segata N},
\textbf{Sinha R}, \textbf{Ulrich CM}, \textbf{Brenner H},
\textbf{Arumugam M}, \textbf{Bork P}, \textbf{Zeller G}. 2019.
Meta-analysis of fecal metagenomes reveals global microbial signatures
that are specific for colorectal cancer. Nature Medicine
\textbf{25}:679.
doi:\href{https://doi.org/10.1038/s41591-019-0406-6}{10.1038/s41591-019-0406-6}.

\leavevmode\hypertarget{ref-vangay_microbiome_2019}{}%
24. \textbf{Vangay P}, \textbf{Hillmann BM}, \textbf{Knights D}. 2019.
Microbiome learning repo (ML repo): A public repository of microbiome
regression and classification tasks. Gigascience \textbf{8}.
doi:\href{https://doi.org/10.1093/gigascience/giz042}{10.1093/gigascience/giz042}.

\leavevmode\hypertarget{ref-galkin_human_2018}{}%
25. \textbf{Galkin F}, \textbf{Aliper A}, \textbf{Putin E},
\textbf{Kuznetsov I}, \textbf{Gladyshev VN}, \textbf{Zhavoronkov A}.
2018. Human microbiome aging clocks based on deep learning and tandem of
permutation feature importance and accumulated local effects. bioRxiv.
doi:\href{https://doi.org/10.1101/507780}{10.1101/507780}.

\leavevmode\hypertarget{ref-reiman_using_2017}{}%
26. \textbf{Reiman D}, \textbf{Metwally A}, \textbf{Dai Y}. 2017. Using
convolutional neural networks to explore the microbiome, pp. 4269--4272.
\emph{In} 2017 39th annual international conference of the IEEE
engineering in medicine and biology society (EMBC).

\leavevmode\hypertarget{ref-fioravanti_phylogenetic_2017}{}%
27. \textbf{Fioravanti D}, \textbf{Giarratano Y}, \textbf{Maggio V},
\textbf{Agostinelli C}, \textbf{Chierici M}, \textbf{Jurman G},
\textbf{Furlanello C}. 2017. Phylogenetic convolutional neural networks
in metagenomics. arXiv:170902268 {[}cs, q-bio{]}.

\leavevmode\hypertarget{ref-thomas_metagenomic_2019}{}%
28. \textbf{Thomas AM}, \textbf{Manghi P}, \textbf{Asnicar F},
\textbf{Pasolli E}, \textbf{Armanini F}, \textbf{Zolfo M},
\textbf{Beghini F}, \textbf{Manara S}, \textbf{Karcher N}, \textbf{Pozzi
C}, \textbf{Gandini S}, \textbf{Serrano D}, \textbf{Tarallo S},
\textbf{Francavilla A}, \textbf{Gallo G}, \textbf{Trompetto M},
\textbf{Ferrero G}, \textbf{Mizutani S}, \textbf{Shiroma H},
\textbf{Shiba S}, \textbf{Shibata T}, \textbf{Yachida S}, \textbf{Yamada
T}, \textbf{Wirbel J}, \textbf{Schrotz-King P}, \textbf{Ulrich CM},
\textbf{Brenner H}, \textbf{Arumugam M}, \textbf{Bork P}, \textbf{Zeller
G}, \textbf{Cordero F}, \textbf{Dias-Neto E}, \textbf{Setubal JC},
\textbf{Tett A}, \textbf{Pardini B}, \textbf{Rescigno M},
\textbf{Waldron L}, \textbf{Naccarati A}, \textbf{Segata N}. 2019.
Metagenomic analysis of colorectal cancer datasets identifies
cross-cohort microbial diagnostic signatures and a link with choline
degradation. Nature Medicine \textbf{25}:667.
doi:\href{https://doi.org/10.1038/s41591-019-0405-7}{10.1038/s41591-019-0405-7}.

\leavevmode\hypertarget{ref-rudin_please_2018}{}%
29. \textbf{Rudin C}. 2018. Please stop explaining black box models for
high stakes decisions. arXiv:181110154 {[}cs, stat{]}.

\leavevmode\hypertarget{ref-rudin_optimized_2018}{}%
30. \textbf{Rudin C}, \textbf{Ustun B}. 2018. Optimized scoring systems:
Toward trust in machine learning for healthcare and criminal justice.
Interfaces \textbf{48}:449--466.
doi:\href{https://doi.org/10.1287/inte.2018.0957}{10.1287/inte.2018.0957}.

\leavevmode\hypertarget{ref-Quinn847475}{}%
31. \textbf{Quinn TP}, \textbf{Erb I}. 2019. Another look at
microbemetabolite interactions: How scale invariant correlations can
outperform a neural network. bioRxiv.
doi:\href{https://doi.org/10.1101/847475}{10.1101/847475}.

\leavevmode\hypertarget{ref-knights_human-associated_2011}{}%
32. \textbf{Knights D}, \textbf{Parfrey LW}, \textbf{Zaneveld J},
\textbf{Lozupone C}, \textbf{Knight R}. 2011. Human-associated microbial
signatures: Examining their predictive value. Cell Host Microbe
\textbf{10}:292--296.
doi:\href{https://doi.org/10.1016/j.chom.2011.09.003}{10.1016/j.chom.2011.09.003}.

\leavevmode\hypertarget{ref-miller_explanation_2017}{}%
33. \textbf{Miller T}. 2017. Explanation in artificial intelligence:
Insights from the social sciences. arXiv:170607269 {[}cs{]}.

\leavevmode\hypertarget{ref-ribeiro_why_2016}{}%
34. \textbf{Ribeiro MT}, \textbf{Singh S}, \textbf{Guestrin C}. 2016.
"Why should i trust you?": Explaining the predictions of any classifier.
arXiv:160204938 {[}cs, stat{]}.

\leavevmode\hypertarget{ref-nori_interpretml:_2019}{}%
35. \textbf{Nori H}, \textbf{Jenkins S}, \textbf{Koch P},
\textbf{Caruana R}. 2019. InterpretML: A unified framework for machine
learning interpretability. arXiv:190909223 {[}cs, stat{]}.

\leavevmode\hypertarget{ref-10.1093ux2fbioinformaticsux2fbtq134}{}%
36. \textbf{Altmann A}, \textbf{Toloşi L}, \textbf{Sander O},
\textbf{Lengauer T}. 2010. Permutation importance: a corrected feature
importance measure. Bioinformatics \textbf{26}:1340--1347.
doi:\href{https://doi.org/10.1093/bioinformatics/btq134}{10.1093/bioinformatics/btq134}.

\leavevmode\hypertarget{ref-breiman_statistical_2001}{}%
37. \textbf{Breiman L}. 2001. Statistical modeling: The two cultures
(with comments and a rejoinder by the author). Statist Sci
\textbf{16}:199--231.
doi:\href{https://doi.org/10.1214/ss/1009213726}{10.1214/ss/1009213726}.

\leavevmode\hypertarget{ref-strobl_bias_2007}{}%
38. \textbf{Strobl C}, \textbf{Boulesteix A-L}, \textbf{Zeileis A},
\textbf{Hothorn T}. 2007. Bias in random forest variable importance
measures: Illustrations, sources and a solution. BMC Bioinformatics
\textbf{8}:25.
doi:\href{https://doi.org/10.1186/1471-2105-8-25}{10.1186/1471-2105-8-25}.

\leavevmode\hypertarget{ref-dormann_collinearity:_2013}{}%
39. \textbf{Dormann CF}, \textbf{Elith J}, \textbf{Bacher S},
\textbf{Buchmann C}, \textbf{Carl G}, \textbf{Carré G}, \textbf{Marquéz
JRG}, \textbf{Gruber B}, \textbf{Lafourcade B}, \textbf{Leitão PJ},
\textbf{Münkemüller T}, \textbf{McClean C}, \textbf{Osborne PE},
\textbf{Reineking B}, \textbf{Schröder B}, \textbf{Skidmore AK},
\textbf{Zurell D}, \textbf{Lautenbach S}. 2013. Collinearity: A review
of methods to deal with it and a simulation study evaluating their
performance. Ecography \textbf{36}:27--46.
doi:\href{https://doi.org/10.1111/j.1600-0587.2012.07348.x}{10.1111/j.1600-0587.2012.07348.x}.

\leavevmode\hypertarget{ref-li_accurate_2020}{}%
40. \textbf{Li J}, \textbf{Liu L}, \textbf{Le TD}, \textbf{Liu J}. 2020.
Accurate data-driven prediction does not mean high reproducibility. Nat
Mach Intell \textbf{2}:13--15.
doi:\href{https://doi.org/10.1038/s42256-019-0140-2}{10.1038/s42256-019-0140-2}.

\leavevmode\hypertarget{ref-sze_fecal_2019}{}%
41. \textbf{Sze MA}, \textbf{Topçuoğlu BD}, \textbf{Lesniak NA},
\textbf{Ruffin MT}, \textbf{Schloss PD}. 2019. Fecal short-chain fatty
acids are not predictive of colonic tumor status and cannot be predicted
based on bacterial community structure. mBio \textbf{10}:e01454--19.
doi:\href{https://doi.org/10.1128/mBio.01454-19}{10.1128/mBio.01454-19}.

\leavevmode\hypertarget{ref-kocheturov_massive_2019}{}%
42. \textbf{Kocheturov A}, \textbf{Pardalos PM}, \textbf{Karakitsiou A}.
2019. Massive datasets and machine learning for computational
biomedicine: Trends and challenges. Ann Oper Res \textbf{276}:5--34.
doi:\href{https://doi.org/10.1007/s10479-018-2891-2}{10.1007/s10479-018-2891-2}.

\leavevmode\hypertarget{ref-kim_improved_2018}{}%
43. \textbf{Kim M}, \textbf{Oh I}, \textbf{Ahn J}. 2018. An improved
method for prediction of cancer prognosis by network learning. Genes
\textbf{9}:478.
doi:\href{https://doi.org/10.3390/genes9100478}{10.3390/genes9100478}.

\leavevmode\hypertarget{ref-wiens_no_2019}{}%
44. \textbf{Wiens J}, \textbf{Saria S}, \textbf{Sendak M},
\textbf{Ghassemi M}, \textbf{Liu VX}, \textbf{Doshi-Velez F},
\textbf{Jung K}, \textbf{Heller K}, \textbf{Kale D}, \textbf{Saeed M},
\textbf{Ossorio PN}, \textbf{Thadaney-Israni S}, \textbf{Goldenberg A}.
2019. Do no harm: A roadmap for responsible machine learning for health
care. Nat Med \textbf{25}:1337--1340.
doi:\href{https://doi.org/10.1038/s41591-019-0548-6}{10.1038/s41591-019-0548-6}.

\leavevmode\hypertarget{ref-westcott_opticlust_2017}{}%
45. \textbf{Westcott SL}, \textbf{Schloss PD}. 2017. OptiClust, an
Improved Method for Assigning Amplicon-Based Sequence Data to
Operational Taxonomic Units. mSphere \textbf{2}.
doi:\href{https://doi.org/10.1128/mSphereDirect.00073-17}{10.1128/mSphereDirect.00073-17}.

\leavevmode\hypertarget{ref-redwood_stool_2016}{}%
46. \textbf{Redwood DG}, \textbf{Asay ED}, \textbf{Blake ID},
\textbf{Sacco PE}, \textbf{Christensen CM}, \textbf{Sacco FD},
\textbf{Tiesinga JJ}, \textbf{Devens ME}, \textbf{Alberts SR},
\textbf{Mahoney DW}, \textbf{Yab TC}, \textbf{Foote PH}, \textbf{Smyrk
TC}, \textbf{Provost EM}, \textbf{Ahlquist DA}. 2016. Stool DNA testing
for screening detection of colorectal neoplasia in alaska native people.
Mayo Clin Proc \textbf{91}:61--70.
doi:\href{https://doi.org/10.1016/j.mayocp.2015.10.008}{10.1016/j.mayocp.2015.10.008}.

\leavevmode\hypertarget{ref-sze_leveraging_2018}{}%
47. \textbf{Sze MA}, \textbf{Schloss PD}. 2018. Leveraging existing 16S
rRNA gene surveys to identify reproducible biomarkers in individuals
with colorectal tumors. mBio \textbf{9}:e00630--18.
doi:\href{https://doi.org/10.1128/mBio.00630-18}{10.1128/mBio.00630-18}.

\leavevmode\hypertarget{ref-schloss_introducing_2009}{}%
48. \textbf{Schloss PD}, \textbf{Westcott SL}, \textbf{Ryabin T},
\textbf{Hall JR}, \textbf{Hartmann M}, \textbf{Hollister EB},
\textbf{Lesniewski RA}, \textbf{Oakley BB}, \textbf{Parks DH},
\textbf{Robinson CJ}, \textbf{Sahl JW}, \textbf{Stres B},
\textbf{Thallinger GG}, \textbf{Van Horn DJ}, \textbf{Weber CF}. 2009.
Introducing mothur: Open-Source, Platform-Independent,
Community-Supported Software for Describing and Comparing Microbial
Communities. ApplEnvironMicrobiol \textbf{75}:7537--7541.

\leavevmode\hypertarget{ref-rognes_vsearch_2016}{}%
49. \textbf{Rognes T}, \textbf{Flouri T}, \textbf{Nichols B},
\textbf{Quince C}, \textbf{Mahé F}. 2016. VSEARCH: A versatile open
source tool for metagenomics. PeerJ \textbf{4}:e2584.
doi:\href{https://doi.org/10.7717/peerj.2584}{10.7717/peerj.2584}.

\leavevmode\hypertarget{ref-li_hyperband:_2016}{}%
50. \textbf{Li L}, \textbf{Jamieson K}, \textbf{DeSalvo G},
\textbf{Rostamizadeh A}, \textbf{Talwalkar A}. 2016. Hyperband: A novel
bandit-based approach to hyperparameter optimization. arXiv:160306560
{[}cs, stat{]}.

\newpage

\captionsetup{labelformat=empty}
\captionof{table}{\textbf{Table 1.} Characteristics of the machine learning models in our comparative study.}
\small
\begin{tabular}{|l|l|l|}
\hline

\rowcolor{lightgray}
\textbf{Model} & \textbf{Description} & \textbf{Linearity} \\ \hline

\makecell[l]{Logistic \\regression} & \makecell[l]{A predictive regression analysis when the dependent \\variable is binary.} & Linear \\ \hline

\makecell[l]{SVM with \\linear kernel} & \makecell[l]{A classifier that is defined by an optimal linear \\separating hyperplane that discriminates between labels.} & Linear  \\ \hline

\makecell[l]{SVM with \\radial basis kernel} & \makecell[l]{A classifier that is defined by an optimal non-linear \\separating hyperplane that discriminates between labels.} & Non-linear \\ \hline

\makecell[l]{Decision tree} & \makecell[l]{A classifier that sorts samples down from the
root to the \\leaf node where an attribute is tested to discriminate \\between labels.} & Non-linear \\ \hline

Random forest & \makecell[l]{A classifier that is an ensemble of decision trees \\ that grows randomly with subsampled data.} & Non-linear \\ \hline

\makecell[l]{Gradient Boosted Trees \\ (XGBoost)} & \makecell[l]{A classifier that is an ensembe of decision trees \\ that grows greedily.} & Non-linear \\ \hline

\end{tabular}
\newpage

\textbf{Figure 1. Machine learning pipeline.} We split the data to
create a training (80\%) and held-out test set (20\%). The splits were
stratified to maintain the overall class distribution. We performed
five-fold cross-validation on the training data to select the best
hyperparameter setting and then used these hyperparameters to train the
models. The model was evaluated on the held-out data set. Abbreviations:
cvAUC, cross-validation area under the receiver operating characteristic
curve.

\hfill\break

\textbf{Figure 2. Generalization and classification performance of ML
models using AUROC values of all cross-validation and testing
performances.} The median AUROC for diagnosing individuals with SRN
using bacterial abundances was higher than chance (depicted by a
horizontal line at 0.50) for all the ML models. The predictive
performance of random forest model was higher than other ML models,
though not significantly (p \textgreater{} 0.05). L2-regularized
logistic regression, XGBoost, L2-regularized SVM with linear and radial
basis function kernel performances were not significantly different from
one another. The boxplot shows quartiles at the box ends and the median
as the horizontal line in the box. The whiskers show the farthest points
that were not outliers. Outliers were defined as those data points that
are not within 1.5 times the interquartile ranges.

\hfill\break

\textbf{Figure 3. Interpretation of the linear ML models.} The ranks of
absolute feature weights of (A) L1-regularized SVM with linear kernel,
(B) L2-regularized SVM with linear kernel, and (C) L2-regularized
logistic regression, were ranked from highest rank, 1, to lowest rank,
100, for each data-split. The feature ranks of the 20 highest ranked
OTUs based on their median ranks (median shown in black) are reported
here. OTUs that were associated with classifying a subject as being
healthy had negative signs and were shown in blue. OTUs that were
associated with classifying a subject having an SRN had positive signs
and were shown in red.

\hfill\break

\textbf{Figure 4. Interpretation of the non-linear ML models.} (A) SVM
with radial basis kernel, (B) decision tree, (C) random forest, and (D)
XGBoost feature importances were explained using permutation importance
on the held-out test data set. The gray rectangle and the dashed line
show the IQR range and median of the base testing AUROC without any
permutation. The 20 OTUs that caused the largest decrease in the AUROC
when permuted are reported here. The colors of the box plots represent
the OTUs that were shared among the different models; yellow were OTUs
that were shared among all the non-linear models, green were OTUs that
were shared among the tree-based models, green were the OTUs shared
among SVM with radial basis kernel, decision tree and XGBoost, pink were
the OTUs shared among SVM with radial basis kernel and XGBoost only, red
were the OTUs shared among random forest and XGBoost only and blue were
the OTUs shared among decision tree and random forest only. For all of
the tree-based models, a \emph{Peptostreptococcus} species (OTU00367)
had the largest impact on predictive performance.

\hfill\break

\textbf{Figure 5. Training times of seven ML models.} The median
training time was the highest for XGBoost and shortest for
L2-regularized logistic regression.

\newpage

\textbf{Figure S1. NMDS ordination of Bray-Curtis distances.} NMDS
ordination relating the community structures of the fecal microbiota
from 490 patients (261 patients with normal colonoscopies and 229
patients who have screen relevant neoplasias; SRNs).

\hfill\break

\textbf{Figure S2. Hyperparameter setting performances for linear
models.} (A) L2-regularized logistic regression, (B) L1-regularized SVM
with linear kernel, and (C) L2-regularized SVM with linear kernel mean
cross-validation AUROC values when different hyperparameters were used
in training the model. The stars represent the highest performing
hyperparameter setting for each model.

\hfill\break

\textbf{Figure S3. Hyperparameter setting performances for non-linear
models.} (A) Decision tree, (B) random forest, (C) SVM with radial basis
kernel, and (D) XGBoost mean cross-validation AUROC values when
different hyperparameters were used in training the model. The stars
represent the highest performing hyperparameter setting for the models.

\hfill\break

\textbf{Figure S4. Histogram of AUROC differences between L2-regularized
logistic regression and random forest for each of the hundred
data-splits.} In 75\% of data-splits, the AUROC of random forest was
greater than that of L2-regularized logistic regression. The p-value was
manually calculated using the sampling distribution of the test
statistic under the null hypothesis. We tested how often random forest
performed more accurately than L2-regularized logistic regression. The
null hypothesis is that the distribution of the difference between the
AUROC values of random forest and L2 logistic regression is symmetric
about 0, therefore the p-value was calculated for a double-tail event.

\hfill\break

\textbf{Figure S5. Classification performance of ML models across
cross-validation when trained on a subset of the dataset.} (A)
L2-regularized logistic regression and (B) random forest models were
trained using the original study design with 490 subjects and subsets of
it with 15, 30, 60, 120, and 245 subjects. The range among the
cross-validation AUROC values within both models at smaller sample sizes
were much larger than when the full collection of samples was used to
train and validate the models but included the ranges observed with the
more complete datasets.

\hfill\break

\textbf{Figure S6. Permutation importance analysis.} Permutation
importance analysis measures the decrease in the predictive performance
of the model after we permute (A) a feature's or (B) a group of
correlated features' values, which breaks the relationship between the
feature and the diagnosis.

\hfill\break

\textbf{Figure S7. Interpretation of the linear ML models with
permutation importance.} (A) L1-regularized SVM with linear kernel, (B)
L2-regularized SVM with linear kernel, and (C) L2-regularized logistic
regression were interpreted using permutation importance using held-out
test set.

\hfill\break

\textbf{Figure S8. Relative abundances of the 20 most important OTUs in
L2-regularized logistic regression and random forest models.} The most
important 20 OTUs were chosen for (A) Random forest and (B)
L2-regularized logistic regression models by permutation importance and
ranking feature coefficients, respectively. The relative abundances of
these OTUs were compared based on the diagnosis of the patients. The
minimal differences betweeen relative abundances for these OTUs show
that it is not possible to differentiate disease vs healthy states by
focusing on individual taxa.

\end{document}
