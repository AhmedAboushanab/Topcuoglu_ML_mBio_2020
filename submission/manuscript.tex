\documentclass[11pt,]{article}
\usepackage{lmodern}
\usepackage{amssymb,amsmath}
\usepackage{ifxetex,ifluatex}
\usepackage{fixltx2e} % provides \textsubscript
\ifnum 0\ifxetex 1\fi\ifluatex 1\fi=0 % if pdftex
  \usepackage[T1]{fontenc}
  \usepackage[utf8]{inputenc}
\else % if luatex or xelatex
  \ifxetex
    \usepackage{mathspec}
  \else
    \usepackage{fontspec}
  \fi
  \defaultfontfeatures{Ligatures=TeX,Scale=MatchLowercase}
\fi
% use upquote if available, for straight quotes in verbatim environments
\IfFileExists{upquote.sty}{\usepackage{upquote}}{}
% use microtype if available
\IfFileExists{microtype.sty}{%
\usepackage{microtype}
\UseMicrotypeSet[protrusion]{basicmath} % disable protrusion for tt fonts
}{}
\usepackage[margin=1.0in]{geometry}
\usepackage{hyperref}
\hypersetup{unicode=true,
            pdftitle={Evaluation of machine learning methods for 16S rRNA gene data},
            pdfborder={0 0 0},
            breaklinks=true}
\urlstyle{same}  % don't use monospace font for urls
\usepackage{graphicx,grffile}
\makeatletter
\def\maxwidth{\ifdim\Gin@nat@width>\linewidth\linewidth\else\Gin@nat@width\fi}
\def\maxheight{\ifdim\Gin@nat@height>\textheight\textheight\else\Gin@nat@height\fi}
\makeatother
% Scale images if necessary, so that they will not overflow the page
% margins by default, and it is still possible to overwrite the defaults
% using explicit options in \includegraphics[width, height, ...]{}
\setkeys{Gin}{width=\maxwidth,height=\maxheight,keepaspectratio}
\IfFileExists{parskip.sty}{%
\usepackage{parskip}
}{% else
\setlength{\parindent}{0pt}
\setlength{\parskip}{6pt plus 2pt minus 1pt}
}
\setlength{\emergencystretch}{3em}  % prevent overfull lines
\providecommand{\tightlist}{%
  \setlength{\itemsep}{0pt}\setlength{\parskip}{0pt}}
\setcounter{secnumdepth}{0}
% Redefines (sub)paragraphs to behave more like sections
\ifx\paragraph\undefined\else
\let\oldparagraph\paragraph
\renewcommand{\paragraph}[1]{\oldparagraph{#1}\mbox{}}
\fi
\ifx\subparagraph\undefined\else
\let\oldsubparagraph\subparagraph
\renewcommand{\subparagraph}[1]{\oldsubparagraph{#1}\mbox{}}
\fi

%%% Use protect on footnotes to avoid problems with footnotes in titles
\let\rmarkdownfootnote\footnote%
\def\footnote{\protect\rmarkdownfootnote}

%%% Change title format to be more compact
\usepackage{titling}

% Create subtitle command for use in maketitle
\newcommand{\subtitle}[1]{
  \posttitle{
    \begin{center}\large#1\end{center}
    }
}

\setlength{\droptitle}{-2em}

  \title{\textbf{Evaluation of machine learning methods for 16S rRNA gene data}}
    \pretitle{\vspace{\droptitle}\centering\huge}
  \posttitle{\par}
    \author{}
    \preauthor{}\postauthor{}
    \date{}
    \predate{}\postdate{}
  
\usepackage{booktabs}
\usepackage{longtable}
\usepackage{array}
\usepackage{multirow}
\usepackage[table]{xcolor}
\usepackage{wrapfig}
\usepackage{float}
\usepackage{colortbl}
\usepackage{pdflscape}
\usepackage{tabu}
\usepackage{threeparttable}
\usepackage{threeparttablex}
\usepackage[normalem]{ulem}
\usepackage{makecell}

\usepackage{helvet} % Helvetica font
\renewcommand*\familydefault{\sfdefault} % Use the sans serif version of the font
\usepackage[T1]{fontenc}

\usepackage[none]{hyphenat}

\usepackage{setspace}
\doublespacing
\setlength{\parskip}{1em}

\usepackage{lineno}

\usepackage{pdfpages}
\floatplacement{figure}{H} % Keep the figure up top of the page

\begin{document}
\maketitle

\vspace{35mm}

Running title: Machine learning methods in microbiome studies

\vspace{35mm}

Begüm D. Topçuoğlu\({^1}\), Jenna Wiens\({^2}\), Patrick D.
Schloss\textsuperscript{1\(\dagger\)}

\vspace{40mm}

\(\dagger\) To whom correspondence should be addressed:
\href{mailto:pschloss@umich.edu}{\nolinkurl{pschloss@umich.edu}}

1. Department of Microbiology and Immunology, University of Michigan,
Ann Arbor, MI 48109

2. Department of Computer Science and Engineering, University or
Michigan, Ann Arbor, MI 49109

\newpage

\linenumbers

\subsection{Abstract}\label{abstract}

\newpage

\subsection{Introduction}\label{introduction}

Advances in sequencing technology and decreasing costs of generating 16S
rRNA gene sequences have allowed rapid exploration and taxonomic
characterization of the human associated microbiome. Currently, the
microbiome field is growing at an unprecedented rate and as a result,
there is an increasing demand for reproducible methods to identify
associations between members of the microbiome and human health.
However, this is an undertaking as human associated microbial
communities are remarkably uneven. It is unlikely that a single species
can explain a disease. Instead, subsets of those communities, in
relation to one another and to their host, account for the differences
in the health outcomes. Therefore, researchers have started to explore
the utility of machine learning (ML) models that use microbiota
associated biomarkers to predict human health and to understand the
microbial ecology of the human diseases such as liver cirrhosis,
colorectal cancer, inflammatory bowel diseases (IBD), obesity, and type
2 diabetes (1--11). However, currently the field's use of ML lacks
clarity and consistency on which methods are used and how these methods
are implemented. Moreover, there is a lack of deliberation on why a
particular ML model is utilized. Recently, there is a trend towards
using more complex ML models such as random forest, extreme gradient
boosting and neural networks (11--14) without a discussion on if and how
much model interpretibility is necessary for the study. If we want to
generate reproducible and robust microbiota-associated biomarker models
we need to (1) implement consistent, accessible and transparent machine
learning practices; (2) use ML models that reflect the goal of the study
as it will inform the expectations of model accuracy, complexity,
interpretibility, computational efficiency and scalability.

To shed light on how much task definition and modeling choices can
affect the outcomes of ML studies, we performed an empirical analysis
comparing several different modeling pipelines using the same dataset.
We used a previously published colorectal cancer (CRC) study (3) which
had fecal 16S rRNA gene sequences and human hemoglobin concentrations
from 490 patients. We built ML models using fecal 16S rRNA gene
sequences and human hemoglobin concentrations to predict patients with
normal or screen relevant neoplasias (SRN) disease status. The study had
261 normal and 229 SRN samples. We established modeling pipelines for
L2-regularized logistic regression, L1 and L2 support vector machines
(SVM) with linear and radial basis function kernels, a decision tree,
random forest and XGBoost. These models increase in complexity while
they decrease in interpretability. Our ML pipeline performed 100
data-splits and utilized held-out test data to evaluate generalization
and predicton performance of each ML model. The mean AUROC varied from
0.67 (std ± 0.05) to 0.82 (std ± 0.04). Random Forest and XGBoost had
the highest mean AUROC for detecting SRN. Despite the simplicity, the
L1-regularized linear kernel SVM performed on par with many of the
non-linear models. In terms of computational efficiency, L2 Logistic
Regression trained the fastest (0.105 hours, std ± 0.013), while XGBoost
took the longest (82.47 hours, std ± 17.045). We found that mean
cross-validation and testing AUROC could vary by as much as 0.062, which
highlights the importance of a seperate held-out test set for
evaluation. Aside from evaluating generalization and classification
performances for each of these models, this study established standards
for modeling pipelines of microbiome-associated machine learning models.

\subsection{Results and Discussion}\label{results-and-discussion}

\textbf{The prediction and generalization performance of classifiers
during cross-validation and when applied to the held-out test data.}

We evaluated the prediction performance of seven binary classification
models when applied to held-out test data using AUROC as our metric.
Random Forest and XGBoost had the highest mean AUROC for detecting SRN,
0.816 (std ± 0.039) and 0.814 (std ± 0.038) respectively {[}Figure 2{]}.
L1 linear SVM and decision tree had significantly lower AUROC values,
0.748 (std ± 0.043) and 0.741 (std ± 0.038) {[}Figure 2{]}. However,
they had significanly higher performances than L2 linear SVM, RBF SVM
and L2 logistic regression which had mean AUROC values of 0.67 (std ±
0.054), 0.684 (std ± 0.049) and 0.677 (std ± 0.053) respectively
{[}Figure 2{]}. We also evaluated the generalization performance of each
classifier by comparing their mean cross-validation AUROC and mean
testing AUROC. We found a statistically significant difference for
classifiers L2 support vector machines (SVM) with linear and radial
basis function kernels and L2 logistic regression. These differences
were 0.062, 0.046 and 0.061, respectively {[}Figure 2{]} .

\textbf{Hyperparameter optimization for each classifier.}

\textbf{The interpretation of classifiers.}

\subsection{Conclusions}\label{conclusions}

\subsection{Materials and Methods}\label{materials-and-methods}

\paragraph{Data collection}\label{data-collection}

The data used for this analysis are stool bacterial abundances, stool
hemoglobin levels and clinical information of the patients recruited by
Great Lakes-New England Early Detection Research Network study. These
data were obtained from Sze et al (15). The stool samples were provided
by recruited adult participants who were undergoing scheduled screening
or surveillance colonoscopy. Colonoscopies were performed and fecal
samples were collected from participants in four locations: Toronto (ON,
Canada), Boston (MA, USA), Houston (TX, USA), and Ann Arbor (MI, USA).
Patients' colonic disease status was defined by colonoscopy with
adequate preparation and tissue histopathology of all resected lesions.
Patients with an adenoma greater than 1 cm, more than three adenomas of
any size, or an adenoma with villous histology were classified as
advanced adenoma. Study had 172 patients with normal colonoscopies, 198
with adenomas and 120 with carcinomas. Of the 198 adenomas, 109 were
identified as advanced adenomas. Stool provided by the patients was used
for Fecal Immunological Tests (FIT) which measure human hemoglobin
concentrations and for 16S rRNA gene sequencing to measure bacterial
population abundances. The bacterial abundance data was generated by Sze
et al, by processing 16S rRNA sequences in Mothur (v1.39.3) using the
default quality filtering methods, identifying and removing chimeric
sequences using VSEARCH and assigning to OTUs at 97\% similarity using
the OptiClust algorithm (16--18).

\paragraph{Data definitions and
pre-processing}\label{data-definitions-and-pre-processing}

The colonic disease status is re-defined as two encompassing classes;
Normal or Screen Relevant Neoplasias (SRNs). Normal class includes
patients with non-advanced adenomas or normal colons whereas SRN class
includes patients with advanced adenomas or carcinomas. Colonic disease
status is the label predicted with each classifier. The bacterial
abundances and FIT results are the features used to predict colonic
disease status. Bacterial abundances are discrete data in the form of
Operational Taxonomic Unit (OTU) counts. There are 6920 OTUs for each
sample. FIT levels are continuous data present for each sample. Because
the data are in different scales, features are transformd by scaling
each feature to a {[}0-1{]} range (Table 1).

\paragraph{Learning the Classifier}\label{learning-the-classifier}

To train and validate our model, labeled data is randomly split 80/20
into a training set and testing set. Then, seven binary class
classifiers, L2 logistic regression, L1 and L2 linear suppor vector
machines (SVM), radial basis function SVM, decision tree, random forest
and XGBoost, are learned. The training set is used for training purposes
and validation of hyperparameter selection, and the test set is used for
evaluation purposes. Hyperparameters are selected using 5-fold
cross-validation with 100-repeats on the training set. Since the colonic
disease status are not uniformly represented in the data, 5-fold splits
are stratified to maintain the overall label distribution on the
training set.

\paragraph{Classifier Performance}\label{classifier-performance}

The classification performance of learned classifier is evaluated on the
labeled held-out testing set. The optimal classifier with optimal
hyperparameters selected in the cross-validation step is used to produce
a prediction for the testing set. The performance of this prediction is
measured in terms of the sensitivity and specificity, in addition to
Area Under the Curve (AUC) metrics. This process of splitting the data,
learning a classifier with cross-validation, and testing the classifier
is repeated on 100 different splits. In the end cross-validation AUC and
testing AUC averaged over the 100 different training/test splits are
reported. Hyperparameter budget and performance for each split is also
reported.

\newpage

\textbf{Figure 1. Generalization and classification performance of
modeling methods } AUC values of all cross validation and testing
performances. The boxplot shows quartiles at the box ends and the
statistical median as the horizontal line in the box. The whiskers show
the farthest points that are not outliers. Outliers are data points that
are not within 3/2 times the interquartile ranges.

\newpage

\subsection{References}\label{references}

\hypertarget{refs}{}
\hypertarget{ref-zeller_potential_2014}{}
1. \textbf{Zeller G}, \textbf{Tap J}, \textbf{Voigt AY},
\textbf{Sunagawa S}, \textbf{Kultima JR}, \textbf{Costea PI},
\textbf{Amiot A}, \textbf{Böhm J}, \textbf{Brunetti F},
\textbf{Habermann N}, \textbf{Hercog R}, \textbf{Koch M},
\textbf{Luciani A}, \textbf{Mende DR}, \textbf{Schneider MA},
\textbf{Schrotz-King P}, \textbf{Tournigand C}, \textbf{Tran Van Nhieu
J}, \textbf{Yamada T}, \textbf{Zimmermann J}, \textbf{Benes V},
\textbf{Kloor M}, \textbf{Ulrich CM}, \textbf{Knebel Doeberitz M von},
\textbf{Sobhani I}, \textbf{Bork P}. 2014. Potential of fecal microbiota
for early-stage detection of colorectal cancer. Mol Syst Biol
\textbf{10}.
doi:\href{https://doi.org/10.15252/msb.20145645}{10.15252/msb.20145645}.

\hypertarget{ref-zackular_human_2014}{}
2. \textbf{Zackular JP}, \textbf{Rogers MAM}, \textbf{Ruffin MT},
\textbf{Schloss PD}. 2014. The human gut microbiome as a screening tool
for colorectal cancer. Cancer Prev Res \textbf{7}:1112--1121.
doi:\href{https://doi.org/10.1158/1940-6207.CAPR-14-0129}{10.1158/1940-6207.CAPR-14-0129}.

\hypertarget{ref-baxter_dna_2016}{}
3. \textbf{Baxter NT}, \textbf{Koumpouras CC}, \textbf{Rogers MAM},
\textbf{Ruffin MT}, \textbf{Schloss PD}. 2016. DNA from fecal
immunochemical test can replace stool for detection of colonic lesions
using a microbiota-based model. Microbiome \textbf{4}.
doi:\href{https://doi.org/10.1186/s40168-016-0205-y}{10.1186/s40168-016-0205-y}.

\hypertarget{ref-baxter_microbiota-based_2016}{}
4. \textbf{Baxter NT}, \textbf{Ruffin MT}, \textbf{Rogers MAM},
\textbf{Schloss PD}. 2016. Microbiota-based model improves the
sensitivity of fecal immunochemical test for detecting colonic lesions.
Genome Medicine \textbf{8}:37.
doi:\href{https://doi.org/10.1186/s13073-016-0290-3}{10.1186/s13073-016-0290-3}.

\hypertarget{ref-hale_shifts_2017}{}
5. \textbf{Hale VL}, \textbf{Chen J}, \textbf{Johnson S},
\textbf{Harrington SC}, \textbf{Yab TC}, \textbf{Smyrk TC},
\textbf{Nelson H}, \textbf{Boardman LA}, \textbf{Druliner BR},
\textbf{Levin TR}, \textbf{Rex DK}, \textbf{Ahnen DJ}, \textbf{Lance P},
\textbf{Ahlquist DA}, \textbf{Chia N}. 2017. Shifts in the fecal
microbiota associated with adenomatous polyps. Cancer Epidemiol
Biomarkers Prev \textbf{26}:85--94.
doi:\href{https://doi.org/10.1158/1055-9965.EPI-16-0337}{10.1158/1055-9965.EPI-16-0337}.

\hypertarget{ref-pasolli_machine_2016}{}
6. \textbf{Pasolli E}, \textbf{Truong DT}, \textbf{Malik F},
\textbf{Waldron L}, \textbf{Segata N}. 2016. Machine learning
meta-analysis of large metagenomic datasets: Tools and biological
insights. PLoS Comput Biol \textbf{12}.
doi:\href{https://doi.org/10.1371/journal.pcbi.1004977}{10.1371/journal.pcbi.1004977}.

\hypertarget{ref-sze_looking_2016}{}
7. \textbf{Sze MA}, \textbf{Schloss PD}. 2016. Looking for a signal in
the noise: Revisiting obesity and the microbiome. mBio \textbf{7}.
doi:\href{https://doi.org/10.1128/mBio.01018-16}{10.1128/mBio.01018-16}.

\hypertarget{ref-walters_meta-analyses_2014}{}
8. \textbf{Walters WA}, \textbf{Xu Z}, \textbf{Knight R}. 2014.
Meta-analyses of human gut microbes associated with obesity and IBD.
FEBS Lett \textbf{588}:4223--4233.
doi:\href{https://doi.org/10.1016/j.febslet.2014.09.039}{10.1016/j.febslet.2014.09.039}.

\hypertarget{ref-vazquez-baeza_guiding_2018}{}
9. \textbf{Vázquez-Baeza Y}, \textbf{Gonzalez A}, \textbf{Xu ZZ},
\textbf{Washburne A}, \textbf{Herfarth HH}, \textbf{Sartor RB},
\textbf{Knight R}. 2018. Guiding longitudinal sampling in IBD cohorts.
Gut \textbf{67}:1743--1745.
doi:\href{https://doi.org/10.1136/gutjnl-2017-315352}{10.1136/gutjnl-2017-315352}.

\hypertarget{ref-qin_alterations_2014}{}
10. \textbf{Qin N}, \textbf{Yang F}, \textbf{Li A}, \textbf{Prifti E},
\textbf{Chen Y}, \textbf{Shao L}, \textbf{Guo J}, \textbf{Le Chatelier
E}, \textbf{Yao J}, \textbf{Wu L}, \textbf{Zhou J}, \textbf{Ni S},
\textbf{Liu L}, \textbf{Pons N}, \textbf{Batto JM}, \textbf{Kennedy SP},
\textbf{Leonard P}, \textbf{Yuan C}, \textbf{Ding W}, \textbf{Chen Y},
\textbf{Hu X}, \textbf{Zheng B}, \textbf{Qian G}, \textbf{Xu W},
\textbf{Ehrlich SD}, \textbf{Zheng S}, \textbf{Li L}. 2014. Alterations
of the human gut microbiome in liver cirrhosis. Nature
\textbf{513}:59--64.
doi:\href{https://doi.org/10.1038/nature13568}{10.1038/nature13568}.

\hypertarget{ref-geman_deep_2018}{}
11. \textbf{Geman O}, \textbf{Chiuchisan I}, \textbf{Covasa M},
\textbf{Doloc C}, \textbf{Milici M-R}, \textbf{Milici L-D}. 2018. Deep
learning tools for human microbiome big data, pp. 265--275. \emph{In}
Balas, VE, Jain, LC, Balas, MM (eds.), Soft computing applications.
Springer International Publishing.

\hypertarget{ref-galkin_human_2018}{}
12. \textbf{Galkin F}, \textbf{Aliper A}, \textbf{Putin E},
\textbf{Kuznetsov I}, \textbf{Gladyshev VN}, \textbf{Zhavoronkov A}.
2018. Human microbiome aging clocks based on deep learning and tandem of
permutation feature importance and accumulated local effects. bioRxiv.
doi:\href{https://doi.org/10.1101/507780}{10.1101/507780}.

\hypertarget{ref-reiman_using_2017}{}
13. \textbf{Reiman D}, \textbf{Metwally A}, \textbf{Dai Y}. 2017. Using
convolutional neural networks to explore the microbiome, pp. 4269--4272.
\emph{In} 2017 39th annual international conference of the IEEE
engineering in medicine and biology society (EMBC).

\hypertarget{ref-fioravanti_phylogenetic_2017}{}
14. \textbf{Fioravanti D}, \textbf{Giarratano Y}, \textbf{Maggio V},
\textbf{Agostinelli C}, \textbf{Chierici M}, \textbf{Jurman G},
\textbf{Furlanello C}. 2017. Phylogenetic convolutional neural networks
in metagenomics. arXiv:170902268 {[}cs, q-bio{]}.

\hypertarget{ref-sze_leveraging_2018}{}
15. \textbf{Sze MA}, \textbf{Schloss PD}. 2018. Leveraging existing 16S
rRNA gene surveys to identify reproducible biomarkers in individuals
with colorectal tumors. mBio \textbf{9}:e00630--18.
doi:\href{https://doi.org/10.1128/mBio.00630-18}{10.1128/mBio.00630-18}.

\hypertarget{ref-schloss_introducing_2009}{}
16. \textbf{Schloss PD}, \textbf{Westcott SL}, \textbf{Ryabin T},
\textbf{Hall JR}, \textbf{Hartmann M}, \textbf{Hollister EB},
\textbf{Lesniewski RA}, \textbf{Oakley BB}, \textbf{Parks DH},
\textbf{Robinson CJ}, \textbf{Sahl JW}, \textbf{Stres B},
\textbf{Thallinger GG}, \textbf{Van Horn DJ}, \textbf{Weber CF}. 2009.
Introducing mothur: Open-Source, Platform-Independent,
Community-Supported Software for Describing and Comparing Microbial
Communities. ApplEnvironMicrobiol \textbf{75}:7537--7541.

\hypertarget{ref-westcott_opticlust_2017}{}
17. \textbf{Westcott SL}, \textbf{Schloss PD}. 2017. OptiClust, an
Improved Method for Assigning Amplicon-Based Sequence Data to
Operational Taxonomic Units. mSphere \textbf{2}.
doi:\href{https://doi.org/10.1128/mSphereDirect.00073-17}{10.1128/mSphereDirect.00073-17}.

\hypertarget{ref-rognes_vsearch_2016}{}
18. \textbf{Rognes T}, \textbf{Flouri T}, \textbf{Nichols B},
\textbf{Quince C}, \textbf{Mahé F}. 2016. VSEARCH: A versatile open
source tool for metagenomics. PeerJ \textbf{4}:e2584.
doi:\href{https://doi.org/10.7717/peerj.2584}{10.7717/peerj.2584}.


\end{document}
