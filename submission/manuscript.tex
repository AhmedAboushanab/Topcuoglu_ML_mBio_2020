\documentclass[11pt,]{article}
\usepackage{lmodern}
\usepackage{amssymb,amsmath}
\usepackage{ifxetex,ifluatex}
\usepackage{fixltx2e} % provides \textsubscript
\ifnum 0\ifxetex 1\fi\ifluatex 1\fi=0 % if pdftex
  \usepackage[T1]{fontenc}
  \usepackage[utf8]{inputenc}
\else % if luatex or xelatex
  \ifxetex
    \usepackage{mathspec}
  \else
    \usepackage{fontspec}
  \fi
  \defaultfontfeatures{Ligatures=TeX,Scale=MatchLowercase}
\fi
% use upquote if available, for straight quotes in verbatim environments
\IfFileExists{upquote.sty}{\usepackage{upquote}}{}
% use microtype if available
\IfFileExists{microtype.sty}{%
\usepackage{microtype}
\UseMicrotypeSet[protrusion]{basicmath} % disable protrusion for tt fonts
}{}
\usepackage[margin=1.0in]{geometry}
\usepackage{hyperref}
\hypersetup{unicode=true,
            pdftitle={Evaluation of binary classification pipelines and methods for 16S rRNA gene data},
            pdfborder={0 0 0},
            breaklinks=true}
\urlstyle{same}  % don't use monospace font for urls
\usepackage{graphicx,grffile}
\makeatletter
\def\maxwidth{\ifdim\Gin@nat@width>\linewidth\linewidth\else\Gin@nat@width\fi}
\def\maxheight{\ifdim\Gin@nat@height>\textheight\textheight\else\Gin@nat@height\fi}
\makeatother
% Scale images if necessary, so that they will not overflow the page
% margins by default, and it is still possible to overwrite the defaults
% using explicit options in \includegraphics[width, height, ...]{}
\setkeys{Gin}{width=\maxwidth,height=\maxheight,keepaspectratio}
\IfFileExists{parskip.sty}{%
\usepackage{parskip}
}{% else
\setlength{\parindent}{0pt}
\setlength{\parskip}{6pt plus 2pt minus 1pt}
}
\setlength{\emergencystretch}{3em}  % prevent overfull lines
\providecommand{\tightlist}{%
  \setlength{\itemsep}{0pt}\setlength{\parskip}{0pt}}
\setcounter{secnumdepth}{0}
% Redefines (sub)paragraphs to behave more like sections
\ifx\paragraph\undefined\else
\let\oldparagraph\paragraph
\renewcommand{\paragraph}[1]{\oldparagraph{#1}\mbox{}}
\fi
\ifx\subparagraph\undefined\else
\let\oldsubparagraph\subparagraph
\renewcommand{\subparagraph}[1]{\oldsubparagraph{#1}\mbox{}}
\fi

%%% Use protect on footnotes to avoid problems with footnotes in titles
\let\rmarkdownfootnote\footnote%
\def\footnote{\protect\rmarkdownfootnote}

%%% Change title format to be more compact
\usepackage{titling}

% Create subtitle command for use in maketitle
\newcommand{\subtitle}[1]{
  \posttitle{
    \begin{center}\large#1\end{center}
    }
}

\setlength{\droptitle}{-2em}

  \title{\textbf{Evaluation of binary classification pipelines and methods for
16S rRNA gene data}}
    \pretitle{\vspace{\droptitle}\centering\huge}
  \posttitle{\par}
    \author{}
    \preauthor{}\postauthor{}
    \date{}
    \predate{}\postdate{}
  
\usepackage{booktabs}
\usepackage{longtable}
\usepackage{array}
\usepackage{multirow}
\usepackage[table]{xcolor}
\usepackage{wrapfig}
\usepackage{float}
\usepackage{colortbl}
\usepackage{pdflscape}
\usepackage{tabu}
\usepackage{threeparttable}
\usepackage{threeparttablex}
\usepackage[normalem]{ulem}
\usepackage{makecell}

\usepackage{helvet} % Helvetica font
\renewcommand*\familydefault{\sfdefault} % Use the sans serif version of the font
\usepackage[T1]{fontenc}

\usepackage[none]{hyphenat}

\usepackage{setspace}
\doublespacing
\setlength{\parskip}{1em}

\usepackage{lineno}

\usepackage{pdfpages}

\begin{document}
\maketitle

\vspace{35mm}

Running title: Machine learning methods in microbiome studies

\vspace{35mm}

Begüm D. Topçuoğlu\({^1}\), Jenna Wiens\({^2}\), Patrick D.
Schloss\textsuperscript{1\(\dagger\)}

\vspace{40mm}

\(\dagger\) To whom correspondence should be addressed:
\href{mailto:pschloss@umich.edu}{\nolinkurl{pschloss@umich.edu}}

1. Department of Microbiology and Immunology, University of Michigan,
Ann Arbor, MI 48109

2. Department of Computer Science and Engineering, University or
Michigan, Ann Arbor, MI 49109

\newpage

\linenumbers

\subsection{Abstract}\label{abstract}

\newpage

\subsection{Introduction}\label{introduction}

As gut microbiome field continues to grow, there will be an
ever-increasing demand for reproducible machine learning methods to
analyze 16S rRNA gene sequence data and to determine association of the
microbiome with a continuous or categorical phenotype of interest. The
use of machine learning in microbiome literature lack clarity over the
learning pipeline which spans the problem formulation, feature
selection, feature pre-processing, model learning, and output. There is
a need for guidance on how to properly implement good machine learning
practices to generate reproducible, robust and actionable models. There
is also a clinical need to generate interpretable models for biomedical
researchers and clinicans to adopt and use regularly (1).

Recently, there is an interest on using machine learning to predict
colorectal cancer progression with microbiota-assiciated biomarkers.
Colorectal cancer is one of the leading cause of death among cancers in
the United States. Each person in the industrialized world has on
average a one-in-twenty chance of developing colorectal cancer (CRC) in
their lifetime and once diagnosed, more than one-third will not survive
5 years (2--4). Colonoscopy as a screening tool is very effective,
however it is very invasive, expensive and have a low rate of patient
adherence. Therefore, there is a need for improved non-invasive methods
to screen individuals. Gut microbiome-based biomarkers can be used as a
non-invasive screening method.

Patients with colorectal cancer have different stool community of
microbes compared to adults with normal colons. This difference however
cannot be explained by a single or a handful of candidate taxa in the
gut microbiome but by many of them in relation to one another.
Therefore, machine learning emerges as a tool to detect the difference
between the gut microbiomes of CRC patients and healthy individuals.
Previous studies have shown that human hemoglobin levels and bacterial
population abundances in the stool help us predict screen relevant
growth in the colon, however the literature for the problem of
classifying colorectal disease status vary greatly, with areas under the
receiver operating characteristic curve (AUC) of 0.7-0.9 (5--8). The
variation in classification performance is based in part on differences
in the task definition, in part on differences in the study populations,
and in part on the learning pipeline. In this study, classification
pipelines with L2-regularized logistic regression, L1 and L2 linear
suppor vector machines (SVM), radial basis function SVM , decision tree,
random forest and XGBoost classifiers are established. The
generalization and prediction performance of these classifiers are
evaluated and each classifier is examined for its reproducibility,
robustness, actionability, interpretibility and susceptibility to
overfitting.

Here, colonic disease status is defined as Normal or Screen Relevant
Neoplasias (SRN). Stool bacterial population abundances and stool
hemoglobin levels of 261 Normal and 229 SRN samples were used to learn
binary classifiers and evaluate their performances. The results show
that \ldots{}\ldots{} (Is there a maximum threshold of prediction with
all these methods? Does an increase in model complexity improve
predictibility?)

\subsection{Results and Discussion}\label{results-and-discussion}

\subparagraph{Results of modeling in text, tables and
figures}\label{results-of-modeling-in-text-tables-and-figures}

\subparagraph{Comparisons among modeling
approaches}\label{comparisons-among-modeling-approaches}

\paragraph{Interpretation of modeling results in terms of
reproducibility, robustness, actionability, interpretibility and
susceptibility}\label{interpretation-of-modeling-results-in-terms-of-reproducibility-robustness-actionability-interpretibility-and-susceptibility}

\paragraph{Consideration of possible weaknesses for each
model}\label{consideration-of-possible-weaknesses-for-each-model}

The interactions between the biomarkers may be nonlinear. Obviously, the
linear models will not incorporate this because they are linear. Tools
like linear models (e.g.~metastats, lefse, wilcoxon, etc) are likely
worthless.

\paragraph{Consideration of possible weaknesses for our approach and
chosen
dataset}\label{consideration-of-possible-weaknesses-for-our-approach-and-chosen-dataset}

\paragraph{Relationship of results to previous literature and broader
implications of this
work}\label{relationship-of-results-to-previous-literature-and-broader-implications-of-this-work}

\paragraph{Prospects of future
progress}\label{prospects-of-future-progress}

\subsection{Conclusions}\label{conclusions}

\subsection{Materials and Methods}\label{materials-and-methods}

\paragraph{Data collection}\label{data-collection}

The data used for this analysis are stool bacterial abundances, stool
hemoglobin levels and clinical information of the patients recruited by
Great Lakes-New England Early Detection Research Network study. These
data were obtained from Sze et al (5). The stool samples were provided
by recruited adult participants who were undergoing scheduled screening
or surveillance colonoscpy. Colonoscopies were performed and fecal
samples were collected from participants in four locations: Toronto (ON,
Canada), Boston (MA, USA), Houston (TX, USA), and Ann Arbor (MI, USA).
Patients' colonic disease status was defined by colonoscopy with
adequate preparation and tissue histopathology of all resected lesions.
Patients with an adenoma greater than 1 cm, more than three adenomas of
any size, or an adenoma with villous histology were classified as
advanced adenoma. Study had 172 patients with normal colonoscopies, 198
with adenomas and 120 with carcinomas. Of the 198 adenomas, 109 were
identified as advanced adenomas. Stool provided by the patients was used
for Fecal Immunological Tests (FIT) which measure human hemoglobin
concentrations and for 16S rRNA gene sequencing to measure bacterial
population abundances. The bacterial abundance data was generated by Sze
et al, by processing 16S rRNA sequences in Mothur (v1.39.3) using the
default quality filtering methods, identifying and removing chimeric
sequences using VSEARCH and assigning to OTUs at 97\% similarity using
the OptiClust algorithm (9--11).

\paragraph{Data definitions and
pre-processing}\label{data-definitions-and-pre-processing}

The colonic disease status is re-defined as two encompassing classes;
Normal or Screen Relevant Neoplasias (SRNs). Normal class includes
patients with non-advanced adenomas or normal colons whereas SRN class
includes patients with advanced adenomas or carcinomas. Colonic disease
status is the label predicted with each classifier. The bacterial
abundances and FIT results are the features used to predict colonic
disease status. Bacterial abundances are discrete data in the form of
Operational Taxonomic Unit (OTU) counts. There are 6920 OTUs for each
sample. FIT levels are continuous data present for each sample. These 2
different data are in different scales. Python programming language
v3.6.6, module scikit-learn v0.19.2 is used to transform features by
scaling each feature to a {[}0-1{]} range (Table 1) (12).

\paragraph{Learning the Classifier}\label{learning-the-classifier}

To train and validate our model, labeled data is randomly split 80/20
into a training set and testing set. Then, seven binary class
classifiers, L2 logistic regression, L1 and L2 linear suppor vector
machines (SVM), radial basis function SVM, decision tree, random forest
and XGBoost, are learned. The training set is used for training purposes
and validation of hyperparameter selection, and the test set is used for
evaluation purposes. Hyperparameters are selected using 5-fold
cross-validation with 100-repeats on the training set. Since the colonic
disease status are not uniformly represented in the data, 5-fold splits
are stratified to maintain the overall label distribution on the
training set. v3.6.6, module scikit-learn v0.19.2 functions are used to
learn the seven classifiers (Table 1).

\paragraph{Classifier Performance}\label{classifier-performance}

The classification performance of learned classifier is evaluated on the
labeled held-out testing set. The optimal classifier with optimal
hyperparameters selected in the cross-validation step is used to produce
a prediction for the testing set. The performance of this prediction is
measured in terms of the sensitivity and specificity, in addition to
Area Under the Curve (AUC) metrics. This process of splitting the data,
learning a classifier with cross-validation, and testing the classifier
is repeated on 100 different splits. In the end cross-validation AUC and
testing AUC averaged over the 100 different training/test splits are
reported. Hyperparameter budget and performance for each split is also
reported.

\newpage

\textbf{Figure 1. Generalization and classification performance of
modeling methods } AUC values of all cross validation and testing
performances. The boxplot shows quartiles at the box ends and the
statistical median as athe horizontal line in the box. The whiskers show
the farthest points that are not outliers. Outliers are data points that
are not within 3/2 times the interquartile ranges.

\newpage

\subsection{References}\label{references}

\hypertarget{refs}{}
\hypertarget{ref-wiens_editorial:_2016}{}
1. \textbf{Wiens J}, \textbf{Wallace BC}. 2016. Editorial: Special issue
on machine learning for health and medicine. Machine Learning
\textbf{102}:305--307.
doi:\href{https://doi.org/10.1007/s10994-015-5533-9}{10.1007/s10994-015-5533-9}.

\hypertarget{ref-seer_2016}{}
2. \textbf{Howlader N KM \textnormal{Noone AM}}. SEER cancer statistics
review, 1975-2013, (national cancer institute. bethesda, md).

\hypertarget{ref-street_colorectal_nodate}{}
3. \textbf{Street W}. Colorectal cancer facts \& figures 2017-2019 40.

\hypertarget{ref-weir_past_2015}{}
4. \textbf{Weir HK}, \textbf{Thompson TD}, \textbf{Soman A},
\textbf{MÞller B}, \textbf{Leadbetter S}. 2015. The past, present, and
future of cancer incidence in the united states: 1975 through 2020.
Cancer \textbf{121}:1827--1837.
doi:\href{https://doi.org/10.1002/cncr.29258}{10.1002/cncr.29258}.

\hypertarget{ref-sze_leveraging_2018}{}
5. \textbf{Sze MA}, \textbf{Schloss PD}. 2018. Leveraging existing 16S
rRNA gene surveys to identify reproducible biomarkers in individuals
with colorectal tumors. mBio \textbf{9}:e00630--18.
doi:\href{https://doi.org/10.1128/mBio.00630-18}{10.1128/mBio.00630-18}.

\hypertarget{ref-baxter_microbiota-based_2016}{}
6. \textbf{Baxter NT}, \textbf{Ruffin MT}, \textbf{Rogers MAM},
\textbf{Schloss PD}. 2016. Microbiota-based model improves the
sensitivity of fecal immunochemical test for detecting colonic lesions.
Genome Medicine \textbf{8}:37.
doi:\href{https://doi.org/10.1186/s13073-016-0290-3}{10.1186/s13073-016-0290-3}.

\hypertarget{ref-baxter_dna_2016}{}
7. \textbf{Baxter NT}, \textbf{Koumpouras CC}, \textbf{Rogers MAM},
\textbf{Ruffin MT}, \textbf{Schloss PD}. 2016. DNA from fecal
immunochemical test can replace stool for detection of colonic lesions
using a microbiota-based model. Microbiome \textbf{4}.
doi:\href{https://doi.org/10.1186/s40168-016-0205-y}{10.1186/s40168-016-0205-y}.

\hypertarget{ref-zackular_human_2014}{}
8. \textbf{Zackular JP}, \textbf{Rogers MAM}, \textbf{Ruffin MT},
\textbf{Schloss PD}. 2014. The human gut microbiome as a screening tool
for colorectal cancer. Cancer Prev Res \textbf{7}:1112--1121.
doi:\href{https://doi.org/10.1158/1940-6207.CAPR-14-0129}{10.1158/1940-6207.CAPR-14-0129}.

\hypertarget{ref-schloss_introducing_2009}{}
9. \textbf{Schloss PD}, \textbf{Westcott SL}, \textbf{Ryabin T},
\textbf{Hall JR}, \textbf{Hartmann M}, \textbf{Hollister EB},
\textbf{Lesniewski RA}, \textbf{Oakley BB}, \textbf{Parks DH},
\textbf{Robinson CJ}, \textbf{Sahl JW}, \textbf{Stres B},
\textbf{Thallinger GG}, \textbf{Van Horn DJ}, \textbf{Weber CF}. 2009.
Introducing mothur: Open-Source, Platform-Independent,
Community-Supported Software for Describing and Comparing Microbial
Communities. ApplEnvironMicrobiol \textbf{75}:7537--7541.

\hypertarget{ref-westcott_opticlust_2017}{}
10. \textbf{Westcott SL}, \textbf{Schloss PD}. 2017. OptiClust, an
Improved Method for Assigning Amplicon-Based Sequence Data to
Operational Taxonomic Units. mSphere \textbf{2}.
doi:\href{https://doi.org/10.1128/mSphereDirect.00073-17}{10.1128/mSphereDirect.00073-17}.

\hypertarget{ref-rognes_vsearch_2016}{}
11. \textbf{Rognes T}, \textbf{Flouri T}, \textbf{Nichols B},
\textbf{Quince C}, \textbf{Mahé F}. 2016. VSEARCH: A versatile open
source tool for metagenomics. PeerJ \textbf{4}:e2584.
doi:\href{https://doi.org/10.7717/peerj.2584}{10.7717/peerj.2584}.

\hypertarget{ref-scikit-learn}{}
12. \textbf{Pedregosa F}, \textbf{Varoquaux G}, \textbf{Gramfort A},
\textbf{Michel V}, \textbf{Thirion B}, \textbf{Grisel O},
\textbf{Blondel M}, \textbf{Prettenhofer P}, \textbf{Weiss R},
\textbf{Dubourg V}, \textbf{Vanderplas J}, \textbf{Passos A},
\textbf{Cournapeau D}, \textbf{Brucher M}, \textbf{Perrot M},
\textbf{Duchesnay E}. 2011. Scikit-learn: Machine learning in Python.
Journal of Machine Learning Research \textbf{12}:2825--2830.


\end{document}
